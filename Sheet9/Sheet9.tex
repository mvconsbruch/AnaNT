\documentclass[a4paper,11pt]{article}
\pagenumbering{arabic}
\usepackage{../environment}

\author{Max von Consbruch}

\begin{document}

\begin{center}
    \huge{Solutions to Sheet 9.}
\end{center}

\section*{Problem 1\&2}
\section*{Problem 3}
\section*{Problem 4}
Okay, let $c>0$ and let $q$ and $q'$ be two exceptional moduli with zeroes characters
$\chi$, $\chi'$ and real zeroes $\beta, \beta'$ satisfying the condition of the
exercise. Let's compare the assumptions with the statement of (5.12).
\begin{enumerate}
    \item[(A)] We have $1 - \frac c{\log q} < \beta$, and similar for $q'$.
    \item[(5.12)] There is some small $d>0$ (independent of $q$ and $q'$) such
        that we have $\min(\beta, \beta') \leq 1 - \frac d{\log(qq')}$. 
\end{enumerate}
If we assume $q < q'$, we certainly obtain
\[
    1 - \frac c {\log q} < 1 - \frac d {\log(qq')}, \quad \text{i.e.}
    \quad \frac dc < \frac  {\log(qq')}{\log q}, \quad \text{i.e.} \quad
    q' > q^{d/c - 1}.
\]
Thus, any $c < d/3$ does the job. 

Aside: This shows that there are $O(\log \log n)$ exceptional moduli up to $n$. 


\contactend

\end{document}
