\documentclass[a4paper,11pt]{article}
\pagenumbering{arabic}
\usepackage{../environment}

\author{Max von Consbruch}

\begin{document}

\begin{center}
    \huge{Solutions to Sheet 9.}
\end{center}

\section*{Problem 1\&2}

Want to estimate
\[
    S_2(x) \coloneqq \sum_{n \leq x, \Omega(n) = 2} 1.
\]
Write it as 
\[
    S_2(x) = \sum_{p \leq \sqrt x} \sum_{p \leq q \leq x/p} 1 =
    \sum_{p \leq \sqrt x} \left(\pi(x/p) - \pi(p)\right) + O(\sqrt x).
\]
Use PNT, get 
\[
    S_2(x) = \sum_{p \leq \sqrt x} \int_p^{x/p} \frac {\dc t}{\log t} +
    O(x \ec^{-c\sqrt{\log x}} ).
\]
for a constant $c>0$ (not the same as in the PNT).
Concept-wise we are done here, as all is left to do is to do partial summation
with $g(t) = \Li(x/t) - \Li(t)$ as smooth weight, and $a_n$ the indicator
function on primes. Estimating the rest is a bit tedious, but straight-forward:

We have $g(\sqrt x) = 0$ and $-g'(t) = \frac 1 {\log t} + \frac{x}{t^2
\log(x/t)}$. We obtain
\[
    S_2(x) = \sum_{p \leq \sqrt x} g(p) = 
    \int_2^{\sqrt x} \frac{\pi(t)}{\log t} + \frac{\pi(t) x}{t^2 \log(x/t)}
    \dc t.
\]
The integral over $\pi(t)/\log t$ can be dealt with quite quickly. We have 
$\pi(t) \ll \frac t{\log t}$, hence
\[
    \int_2^{\sqrt x} \frac{\pi(t)}{\log t} \dc t
    \ll \int_2^{\sqrt x} \frac t{(\log t)^2} \dc t
    \ll \left( \int_{2}^{x^{1/4}} + 
    \int_{x^{1/4}}^{\sqrt x} \right )\frac{t}{(\log t)^2} \dc t 
    \ll x^{1/2} + \frac x{(\log x)^2}.
\]
We are left with
\[
    S_2(x) = \int_2^{\sqrt x} \frac{\pi(t) x \dc t}{t^2 \log(x/t)} + O\left(\frac x
    {(\log x)^2}\right) = \int_2^{\sqrt x} \frac{x((\log t)^{-1} + O((\log t)^{-2}))}{t
\log(x/t)} \dc t + O\left(\frac x{(\log x)^2}\right),
\]
where we applied the PNT again, this time with error term $O(x/(\log x)^2)$. 
The integral over the $O$-term is also easily handled. We have $\log(x/t) \gg
\log g$, and hence find that the contribution is bounded by 
\[
    \frac x{\log x} \int_2^{\sqrt x} \frac 1{t (\log t)^2} \dc t
    \ll \frac x{\log x}. 
\]
We are left with
\[
    S_2(x) = x \int_2^{\sqrt x} \frac 1{t (\log t)(\log \tfrac xt)} \dc t 
    + O(\frac x{\log x}).
\]
We can use the geometric series to show that 
\[
    \frac 1{\log \tfrac xt} = \frac 1{\log x (1- \tfrac{\log x}{\log t})}
    = \frac 1{\log x} \left( 1 + O(\tfrac {\log t}{\log x}) \right)
    = \frac 1 {\log x} + O\left(\frac{\log t}{(\log x)^2} \right).
\]
Hence we obtian
\[
    S_2(x) = \frac x{\log x} \int_2^{\sqrt x} \frac 1{t \log t} \dc t
    + O\left(\frac x{(\log x)^2} \int_2^{\sqrt x} \frac 1t \right)+ O(\tfrac x{\log x}) 
    = \frac x{\log x} \int_2^{\sqrt x} \frac 1{t \log t} \dc t + O\left( \frac
    {x}{\log x} \right).
\]
This integral is exactly given by 
\[
    \int_2^{\sqrt x} \frac 1{t \log t} \dc t = 
    \log \log \sqrt x - \log \log 2,
\]
which leaves us with
\[
    S_2(x) = \frac{x \log \log x}{\log x} + O\left( \frac x{\log x}\right),
\]
as desired.


\section*{Problem 3}
This is a consequence of Merten's theorem, which states that for 
$x > 1$, 
\[
    \sum_{p \leq x} \frac 1p = \log \log x + C + O((\log x)^{-1})
\]
for some constant $C$. 

Note that 
\[
    \frac{\phi(n)}n = \prod_{p \mid n} (1 - p^{-1}),
\]
so we really want to show that the RHS is $\gg (\log \log n)^{-1}$. 
The product over the prime divisors of $n$ is hard to get a hold on. 
It would be much easier if we could somehow relate this to products of 
the form $\prod_{p \leq x} (1-p^{-1})$, as these products can be bounded
with Merten's formula:
\begin{multline*}
    \prod_{p \leq x} \left(1 - \frac 1p \right) 
    = \exp \left( \sum_{p \leq x} \log \left( 1- \frac 1p \right) \right) = \exp \left( - \sum_{p \leq x} \frac 1p - \sum_{p \leq x} \sum_{k \geq 2} \frac1{kp^k} \right) \\ 
    = \exp\left( - \log \log x -C + O((\log x)^{-1}) - \sum_p \sum_{k \geq 2} \frac1{kp^k} + O\left( \sum_{p > x} \sum_{k \geq 2} \frac1 {p^k} \right) \right) \\ 
    = \frac{ \ec^{-C'}}{\log x} \exp \left( O \left( \frac 1{\log x} \right) \right) 
    = \frac{\ec^{-C'}}{\log x}(1 + O((\log x)^{-1}) 
    \gg \frac 1{\log x}.
\end{multline*}
(This also was on sheet 0). 
In particular, if we choose $x = \log n$, we obtain
\[
    \prod_{p \leq \log n}\left(1 - \frac 1p\right) \gg (\log \log n)^{-1}.
\]
This is nice, because the prime divisors $p \mid n$ with $p \geq \log n$ 
don't contribute anything: 
\[
    \prod_{p \mid n, \ p \leq \log n} \left(1 - \frac 1p\right)
    \geq \left( 1 - \frac 1 {\log n} \right)^{\omega(n)}
    \geq \left(1 - \frac 1 {\log n} \right)^{2 \log n} \gg 1.
\]
(Here we used $\omega(n) \leq \log_2(n) \leq 2 \log n$ and that one formula for $\ec$). Hence we can conclude
\[
    \frac{\phi(n)}n \geq
    \left( 1 - \frac 1{\log n} \right)^{\omega(n)} \prod_{p \leq \log n}
    \left(1 - \frac 1p\right) \gg \frac 1{\log \log n}.
\]

\textbf{Notes after correcting.} I just realized that the long calculation 
can be replaced by a reference to (5.9).


\section*{Problem 4}
Okay, let $c>0$ and let $q$ and $q'$ be two exceptional moduli with zeroes characters
$\chi$, $\chi'$ and real zeroes $\beta, \beta'$ satisfying the condition of the
exercise. Let's compare the assumptions with the statement of (5.12).
\begin{enumerate}
    \item[(A)] We have $1 - \frac c{\log q} < \beta$, and similar for $q'$.
    \item[(5.12)] There is some small $d>0$ (independent of $q$ and $q'$) such
        that we have $\min(\beta, \beta') \leq 1 - \frac d{\log(qq')}$. 
\end{enumerate}
If we assume $q < q'$, we certainly obtain
\[
    1 - \frac c {\log q} < 1 - \frac d {\log(qq')}, \quad \text{i.e.}
    \quad \frac dc < \frac  {\log(qq')}{\log q}, \quad \text{i.e.} \quad
    q' > q^{d/c - 1}.
\]
Thus, any $c < d/3$ does the job. 

This shows that there are $O(\log \log n)$ exceptional moduli up to $n$. 

Aside: There is nothing special about the $2$ in the exponent, if we choose 
$c$ small enough we can get arbtirarily large exponents. But gives stronger conditions on what it means to be exceptional. 


\contactend

\end{document}
