\documentclass[a4paper,11pt]{article}
\pagenumbering{arabic}
\usepackage{../environment}


\begin{document}

\begin{center}
    \huge{ Solution to Sheet 3. }
\end{center}

\section*{Facts from multiplicative number theory.} Given some $n = p_1^{e_1} \cdots
p_r^{e_r} \in \N$, we want to investigate the structure of the multiplicative group
$\left ( \Z/n\Z \right )^\times$. By the chinese remainder theorem we find 
\[
    (\Z / n\Z)^\times \cong \left ( \prod_{i = 1}^n (\Z / p_i^{e_i}\Z) \right
        )^\times \cong \prod_{i = 1}^n (\Z / p_i^{e_i}\Z)^\times,
\]
so we really only care about the structure of $(\Z/p^e\Z)^\times$. There, the 
structure is given by
\[
    (\Z/p^e\Z)^\times \cong 
    \begin{cases} 
        \text{a cyclic subgroup of order $\phi(p^e)$} & \text{ if $p$ is odd} \\
        \langle 3 \rangle & \text{ if $p=2$ and $e \leq 2$} \\
        \pm \langle 5 \rangle \cong \Z/2\Z \times \Z/2^{e-2}\Z & \text{ if
        $p=2$ and $e \geq 3$.} 
    \end{cases}
\]
A generator of $\FF_p^\times$, or more generally, a generator of $(\Z/p^e\Z)^\times$
is called a \textit{root of unity}.
We have the \textit{Legendre symbol}, which for $a \in \Z$ and an odd prime $p$
is given by 
\[
    \legendre ap = 
    \begin{cases}
        0 \quad & \text{if } p \mid a\\
        (-1) \quad & \text{if there is no solution mod $p$ to }  x^2 = a\\
        1 \quad & \text{otherwise.}\\
    \end{cases}
\]
It is multiplicative in $a$, hence it yields a character $(\Z/p\Z)^\times \to
\C^\times$. The subgroup of \textit{quadratic residues mod $p$} is given by
$\ker \left( \legendre -p \right) = \langle \varpi^2 \rangle$ for $\varpi$ a
root of unity. \textit{Quadratic reciprocity} states that for two odd primes
$p,q$, we have
\[
    \legendre pq = (-1)^{\frac{p-1}2 \cdot \frac{q-1}2} \legendre qp,
\]
and there are the \textit{supplementary laws}
\[
    \legendre{-1}p = (-1)^{\frac {p-1}2} \quad \text{and} \quad
    \legendre 2p = (-1)^{\frac{p^2-1}8}.
\]

Given a finite abelian group $G$, we define the group of \textit{characters of $G$}
as 
\[
    \widehat G = \Hom_\Ab(G, \C^\times) = \Hom_\Ab(G, S^1).
\]
Given a cyclic group $G \cong \Z/n\Z$, there is an isomorphism $G \cong
\widehat G$ given by $a \mapsto (1 \mapsto \zeta_n^a)$, where $\zeta_n$ is an
$n$-th root of unity. As we also have $\widehat G \oplus \widehat H =
\widehat{G \oplus H}$, this shows that there are isomorphisms $G \cong \widehat
G$ for \textit{all} finite abelian groups\footnote{The first isomorphism is the
universal property of the direct sum: We have $$\Hom_\Ab(G \oplus H, \C^\times)
\cong \Hom_\Ab(G,\C^\times) \oplus \Hom_\Ab(H,\C^\times).$$ Remember that every
finite group is a finite product (equivalently, finite direct sum) of cyclic
groups.}.


\section*{Exercise 1 \& 2.}
\textbf{1.} Note that the real characters are exactly those $\chi:
(\Z/p\Z)^\times \to \C^\times$  with $\chi^2 = 1$.
As $p$ is odd, there are exactly two solutions to $x^2=1$, hence there are exactly
$2$ real characters mod $p$, one of which is the trivial
one (induced by the principle character mod $1$), and the other is given by the 
legendre symbol. The same reasoning goes through mod $p^e$ for $e \geq 2$ (the
multiplicative group is cyclic of even order), but now the characters are
induced from characters mod $p$.

\textbf{2.}
For $n = 2^r$, we find again that the real Dirichlet characters are in
bijection with the set $\{x \in \Z/n\Z \mid x^2 -1 = 0\}$. By the structure of the 
multiplicative group given above, this set has $1$ element if $r=1$, it has $2$
elements if $r = 2$ and $4$ elements if $r \geq 3$. We find:
\begin{itemize}
    \item The multiplicative group of $\Z/2\Z$ is trivial, so there is only the
        character given by $1 \mapsto 1$, which is induced by the principle
        character. 
    \item On $\Z/4\Z$ we have again the principle character and the primitive
        character $\chi_{-4}$ uniquely defined via $\chi_{-4}(-1) = -1$.
    \item On $\Z/8\Z$ we have the principle character, the one induced by $\chi_{-4}$ 
        and the two characters $\chi_{\pm 8}$, where $\chi_{\pm 8}(3) = \mp 1$, 
        $\chi_{\pm 8}(5) = -1$ and $\chi_{\pm 8}(7) = \pm 1$. 
\end{itemize}

\textbf{3.} 
%We first do uniqueness. Assume that we are given two characters $\chi_1$ mod $r$ 
%and $\chi_2$ mod $s$ such that for all $m \in \N$, 
%\[
%    \chi(m) = \chi_1(m \text{ mod $r$}) \chi_2(m \text{ mod $s$}).
%\]
%Then whenever we are given $m \in \N$ such that $m \equiv 1$ mod $s$, we find 
%\[
%    \chi(m) = \chi_1(m),
%\]
%and similarly for $\chi_2$. But the chinese remainder theorem asserts that
%these equalities already define $\chi_1$ and $\chi_2$ uniquely: For any $a \in
%(\Z/r\Z)^\times$, there is some $m \in \N$ such that $m \equiv a \mod r$ and 
%$m \equiv 1 \mod s$. 
%
%Now it is also easy to see existence. For any $a \in (\Z/r\Z)^\times$, simply define
%$\chi_1(a) = \chi(q)$, where $q \in (\Z/n\Z)^\times$ is the unique residue with
%$q \equiv a \pmod r$ and $q \equiv 1 \pmod s$. Do the same for $\chi_2$.
%Multiplicativeness of $\chi_1$ and $\chi_2$ is immediate, and
%by definition we now have $\chi_1 \chi_2 = \chi$. 
We inspect the map 
\[
    \mu: \widehat{(\Z/r\Z)^\times} \times \widehat{(\Z/s\Z)^\times} \to
    \widehat{(\Z/n\Z)^\times} \quad (\chi_1, \chi_2) \mapsto \chi_1\chi_2.
\]
We claim that this map is injective. Indeed, assume that we are given two
characters $\chi_1$ mod $r$ 
and $\chi_2$ mod $s$ such that for all $m \in \N$, 
\[
    \chi(m) = \chi_1(m \text{ mod $r$}) \chi_2(m \text{ mod $s$}).
\]
Then whenever we are given $m \in \N$ such that $m \equiv 1$ mod $s$, we find 
\[
    \chi(m) = \chi_1(m),
\]
and similarly for $\chi_2$. But the chinese remainder theorem asserts that
these equalities already define $\chi_1$ and $\chi_2$ uniquely: For any $a \in
(\Z/r\Z)^\times$, there is some $m \in \N$ such that $m \equiv a \mod r$ and 
$m \equiv 1 \mod s$. Now $\mu$ is an injective map of sets with the same cardinality,
hence bijective.

It remains to show that $\chi_1$ and $\chi_2$ are primitive iff $\chi$ is. Suppose first
that $\chi_1$ was not primitive, i.e., has conductor $d < r$. Then we can write 
$\chi_1 = \Tilde \chi \chi_{0,r}$ where $\Tilde\chi$ is a character mod $d$ and 
$\chi_{0,r}$ is the primitive character mod $r$. Now $\chi' = \Tilde \chi \chi_2$ 
is a character modulo $ds$ and induces $\chi$, since
\[
    \chi = \chi \chi_{0,rs} = \chi_1 \chi_2 \chi_{0,rs} = \Tilde \chi \chi_{0,r}\chi_2
    \chi_{0,rs} = \chi' \chi_{0,r} \chi_{0,rs} = \chi' \chi_{0,rs}.
\]
There is a neat way to now show the converse. Let $\phi_2(n)$ denote the number of 
primitive characters mod $n$. For any $d \mid n$, the set of primitive characters
mod $d$ is in bijection with the characters mod $n$ of conductor $d$, so we find
\[
    \phi(n) = \# \widehat{\left( \Z /n \Z \right)^\times} = \sum_{d \mid n} \phi_2(n) = (1 \star \phi_2)(n),
\]
implying that $\phi_2 = \mu \star \phi$ by moebius-inversion. Hence $\phi_2$ is 
multiplicative. We have shown alrady that the inverse of $\mu$ restricts to a
(necessarily) injective map
\[
    \mu^{-1}: \text{\{primitive characters mod $n$\}} \to \text{\{pr.
    characters mod $r$\}} \times \text{\{pr. characters mod $s$\}}. 
\]
By multiplicity of $\phi_2$, this is a injective map of sets of the same cardinality,
therefore $\mu^{-1}$ is a bijection, and we are done. 

Alternatively we can calculate this directly. Assume that $\chi_1$ and $\chi_2$
are primitive. Choose a character $\Tilde \chi$ mod $d$ that induces $\chi$, so
we may write $$\chi_1\chi_2 = \Tilde \chi \chi_{0,rs} = (\Tilde \chi_1
\chi_{0,r}) (\Tilde \chi_2 \chi_{0,s}),$$ where $\Tilde \chi_1$ is a character
of conducter $d_1 \mid r$ and $\Tilde \chi_2$ is a character of conducter $d_2
\mid s$. But by uniqueness of $\chi_1$ and $\chi_2$, we find $\chi_1 = \Tilde
\chi \chi_{0,r}$ and $\chi_2 = \Tilde \chi \chi_{0,s}$, implying $d = rs$ by
primitivity of $\chi_1$ and $\chi_2$. 

\textbf{4.} Writing $n = 2^r q$ with $q$ odd, we find that the number of primitive real
characters mod $n$ is given by 
\[
    \begin{cases}
        1 \quad &\text{if $r = 0$ and $q$ square-free}, \\
        0 \quad &\text{if $r = 1$ and $q$ square-free}, \\
        1 \quad &\text{if $r = 2$ and $q$ square-free}, \\
        2 \quad &\text{if $r = 3$ and $q$ square-free}, \\
        0 \quad &\text{if $r \geq 4$ or $q$ not square-free}. 
    \end{cases}
\]

\textbf{5.} Clearly the product of two fundamental discriminants (FDs) is again a FD,
and we have $\chi_{D_1 D_2} = \chi_{D_1}\chi_{D_2}$. Also, given a fundamental
discriminant $D$ with $\abs D = d_1 d_2$ and $(d_1, d_2) = 1$, there are
fundamental discriminants $D_1, D_2$ with $d_i = \pm D_i$. 
So we can reduce to the case where
$\abs D = p^r$ is a prime power. As a first reality check, we find that if $p$
is odd,
the only fundamental discriminant of this type is $D = (-1)^{\frac{p-1}2} p$,
in which case $\chi_D$ is given by the unique real primitive character, given by (using quadratic reciprocity)
$$\chi_D(q) = \legendre {(-1)^{(p-1)/2}p}q = \legendre qp.$$
There are no FDs with $\abs D =2$ or $\abs D = 2^r$ with $r \geq 4$. If $\abs D = 4$
there is one ($D = -4$), and if $n = 8$ there are two ($D = \pm 8$).
Using quadratic reciprocity and the supplementary laws, it is easily seen that these
are exactly the characters described above.

\end{document}
