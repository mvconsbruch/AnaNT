\documentclass[a4paper,11pt]{article}
\pagenumbering{arabic}
\usepackage{../environment}

\author{Max von Consbruch}

\begin{document}

\begin{center}
    \huge{Solutions to Sheet 7.}
\end{center}

\section*{Problem 1\&2}
\section*{Problem 3}
\textbf{a)}

\textbf{b)} The good thing with smooth weights is that their mellin transform
vanishes rapidly along vertical lines and we do not have to worry about cutting
off the integral. Perron's formula reveals with $c > 1/2$
\[
    \sum_{n \text{ squarefull}} \ec^{-n/x} = \pifrac \int_{(c)} \frac{\zeta(2s) \zeta(3s)}
    {\zeta(6s)} x^s \Gamma(s) \dc s.
\]
As $\Gamma$ vanishes rapidly along vertical lines, we can shift the contour to 
$\Re s = 1/6 + \varepsilon$ and obtain
\[
    \dots = \frac 12 \frac{\zeta(3/2)}{\zeta(3)} x^{1/2} + 
    \frac 13 \frac{\zeta(2/3)}{\zeta(2)} x^{1/3} + \pifrac \int_{(1/6 + \varepsilon)}
    \frac{\zeta(2s) \zeta(3s)} {\zeta(6s)} x^s \Gamma(s) \dc s.
\]
The integral is absolutely convergent, hence gives an error of size
$O(x^{1/6+\varepsilon})$.

Remark: We will later prove that $\zeta(s)$ does not have zeroes in some
neighbourhood of the line $\Re s = 1$, which in particular implies that there
are no zeroes on the line itself. Hence we can get even shift the contour onto
$\Re s = 1/6$, killing the $+\varepsilon$. 

\section*{Problem 4}
\textbf{a)} Every finite abelian group can be decomposed as a product of 
cyclic groups of prime-power-order. Hence the number of isomorphism classes 
of abelian groups of order $n$ gives a multiplicative arithmetic function
$$a: \N \to \N, \quad n \mapsto \#(\{\text{abelian groups of order
$n$}\}/\cong).$$ 
If $n = p^r$ is a prime power, we find that $a(n)$ is given by the number of 
(additive) partitions of $r$. Indeed, to a partition
\[
    1 \cdot a_1 + 2\cdot a_2 + 3 \cdot a_3 + \dots  = r 
\]
we can associate a group $(\Z/p\Z)^{a_1} \times (\Z/p^{2}\Z)^{a_2} \times
(\Z/p^3\Z)^{a_3} \times \dots$ of order $p^r$, and vice versa. One quickly verifies
(at least formally), that
\[
    \sum_{n=1}^\infty a(n) x^n = (1+x+x^2+\dots) (1+x^2+x^4+\dots) 
    (1+x^3+x^6+\dots)\cdots
\]
and substituting $x=p^{-s}$ for varying $p$ yields the desired formula
\[
    \sum_{n=1}^\infty a(n) n^{-s} = \prod_{p} \prod_{r=1}^\infty(1-p^{-rs})^{-1}
    = \prod_{r=1}^\infty \zeta(rs).
\]
The last step might demand clearification. Remember that a product $\prod a_n$
with $a_n \neq 0$ converges absolutely to something $\neq 0$ iff the sum $\sum
\abs{a_n-1}$ converges absolutely. In $\Re s > 1+\delta$ we have the uniform
bound
\[
    \abs{1-\zeta(s)} \ll \sum_{n=2}^\infty n^{-1-\delta} \ll_\delta 2^{-\delta},
\]
which shows that indeed, the product converges absolutely and locally uniformly in
$\Re s > 1$. 

\textbf{b)} The heuristic goes as follows. Let $F$ be the Dirichlet series attached
to $a$. By the above, $F$ is a holomorphic function for $s > 1$, but writing 
$\zeta(s) = \Delta(s) \zeta(1-s)$ we find that $F$ has a continuation to a 
meromorphic function on $\Re s > 1/2$. (Aside: We can apply the functional equation
to as many $\zeta$-factors as we want, yielding continuations to $\Re s > 1/n$ for 
arbitrarily large $n \in \N$. But $F$ can never be meromorphically continued
to all of $\C$. This is because there are poles at $s=1, 1/2, 1/3, \dots$, which 
by the identity theorem implies that $F^{-1}=0$.) Now Perron's Formula reads
\[
    \sum_{n \leq x} a(n) = \pifrac \int_{(c)} F(s) x^s \frac{\dc s}s,
\]
and upon shifting the contour to $1-\varepsilon$ we obtain
\[
    \sum_{n \leq x}a(n) = x\text{Res}_{s=1} F(s) + \pifrac \int_{(1-\varepsilon)}
    F(s)x^s \frac{\dc s}s.
\]
The residue is given by $C = \zeta(2) \zeta(3) \cdots$, and we'd hope that we would be able to approximate the integral by something of size 
$o(x)$.

\textbf{Proving the asymptotic using Cesàro-weights.} Proving the asymptotic is
quite challenging, as we would have to find some bound on $a(n)$ to apply (4.7).
Further below is a solution assmuing $a(n) \ll n^{1/2+\varepsilon}$, but 
it turns out we don't need such bounds! Instead of bounding 
\[
    S_0(x) = \sum_{n \leq x} a(n),
\]
we choose to bound
\[
    S_1(x) = \sum_{n \leq x} a(n) (x-n) = \int_1^x S_0(y) \dc y.
\]
(These weights are called Cesàro weights). 
Integrating Perron's formula, we find that 
\[
    S_1(x) = \pifrac \int_{(c)} F(s) x^{s+1} \frac{\Gamma(s)}{\Gamma(s+2)} \dc s.
\]
The $\Gamma$-factor is essentially bounded by $\abs s^{-2}$, at least for 
$\abs s > 2$. Whenever $\sigma > 1/2 + \delta$ and $\abs t > 1$ we find  
$$F(\sigma + \ic t) \ll \abs{\zeta(s)} \zeta(1+2\delta) \zeta(3/2+3/2\delta) \cdots
\ll \abs t^{\frac {1-\sigma}2 + \varepsilon} \delta^{-1}.$$
Hence we can shift the contour to $\Re s = 1/2+\delta$, pick up a pole and the 
remaining integral remains absolutely convergent. In formulas,
\[
    S_1(x) = \frac{x^2}2 C + \int_{(1/2+\delta)} F(s) x^{s+1}
    \frac{\Gamma(s)}{\Gamma(s+2)} \dc s = \frac{x^2}2 C +
    O_\delta(x^{3/2+\delta}).
\]
Nice, this at least shows that there is an asymptotic \textit{on average}. But
how can we make use of this? We also showed that the Lindelöf-Hypothesis is true
\textit{on average}, but we are far from proving the Lindelöf-Hypothesis in
general! What plays in our favor here is that $S_0$ is non-decreasing. Denote
by $E_0(x)$ the error function $S_0(x) - Cx$, and define $E_1$ as the integral
of $E_0$. Note that we have $E_1(x) \ll x^{3/2+\varepsilon}$ by the above. 
We also make a choice of some $Q = x^\alpha$ for $\alpha \in [0,1]$ and get
(using monotonicity of $S_0$)
\[
    E_1(x+Q)-E_1(x) = \int_x^{x+Q} E_0(t) \dc t \geq Q(S_0(x) - Cx - CQ) = QE_0(x) + O(Q^2).
\]
But we also know that $E_1(x+Q)-E_1(x) = O(x^{3/2+\varepsilon})$, implying
\[
    QE_0(x) \leq O(x^{3/2+\varepsilon}+Q^2).
\]
This shows $E_0(x) \leq O(x^{3/4+\varepsilon})$ once we choose $Q=x^{3/4}$. A similar
lower bound can be established by inspecting $\int_{x-Q}^x E_0(t) \dc t$ (exercise, 
haha). This proves $S_0(x) = Cx + O(x^{3/4+\varepsilon})$. This really is remarkable,
as this in particular implies that $a(n) \ll n^{3/4 + \varepsilon}$, which is a 
bound we did not know existed beforehand. 


\textbf{Proving the asymptotic assuming $a(n) \ll n^{1/2+\varepsilon}$. (I did
this first but then realized that we can avoid this.)} To prove
    the asymptotic, we first need to get some bound on $a(n)$. Using 
the asymptotic for the number of partitions, one should be able to show that 
$a(n) \ll n^{1/2+\varepsilon}$. (Might think about this again later.) 
Using (4.7), we find for $T = x^\alpha$ (to be chosen later) 
\[
    \sum_{n \leq x} a(n) = \pifrac \int_{(c)}^T F(s) x^s \frac{\dc s}s 
    + O(E),
\]
where $\int_{(c)}^T$ means $\int_{c-\ic T}^{c+\ic T}$ and 
\[
    E = x^{c-\alpha} \sum_n a_n n^{-c} + x^{1/2+\varepsilon}(1+ x^{1-\alpha}\log x)
    \ll_{c} x^{c-\alpha} + x^{1/2+\varepsilon} + x^{3/2 - \alpha + \varepsilon} \log x
\]
For $0< \delta < 1/2$ and $1-\delta < \Re s$ we have a uniform bound $F(s) \ll
(\delta-1/2)^{-1} \zeta(s)$ as the remaining factors are bound by $\zeta(2(1-\delta))
\zeta(3(1-\delta)) \cdots \ll (1/2-\delta)^{-1}$. Hence the convexity bound gives
$$F(\sigma + \ic t) \ll \abs t^{\frac {1-\sigma}2 + \varepsilon}$$
for $\abs t > 1$.
We are now ready for shifting the contour to $\Re s = 1-\delta$. Similarly to above,
we obtain
$$\sum_{n \leq x} = Cx + H(\delta) + \pifrac \int_{(1-\delta)}^T F(s) x^s
\frac{\dc s}s + O(E),$$
where the horizontal integrals are given and bounded by 
\[
    H(\delta) = \pifrac \left( \int_{c-\ic T}^{1-\delta - \ic T} - \int_{c+\ic
    T}^{1-\delta + \ic T} \right) F(s) x^s \frac{\dc s }s \ll (1/2-\delta)^{-1}
    x^c T^{\delta/2-1 + \varepsilon}.
\]
Only the vertical integral remains mysterios. We have $F(s) \ll (\delta-1/2)^{-1} \zeta(s)$, and obtain
\[
    V(\delta) = \pifrac \int_{(1-\delta)^T} F(s) x^s \frac{\dc s }s 
\ll x^{1-\delta}((1/2-\delta)\delta)^{-1} + (1/2-\delta)^{-1}x^{1-\delta}\int_1^T \frac{\abs{\zeta((1-\delta) + \ic t)}}{\abs t} \dc t
\]
We cut the integral in diadic pieces, which we can bound using 
the moment bounds, as
\begin{multline*}
    \int_T^{2T} \frac{\abs{\zeta(\sigma + \ic t)}}{\abs t} \dc t
    \ll \left(\int_T^{2T}\abs{\zeta(\sigma+\ic t)}^2 t^{-1} \dc t \right)^{1/2}
    \left( \int_T^{2T} t^{-1} \dc t\right)^{1/2} \\
    \ll (T^{-1} T^{1+\varepsilon})^{1/2} (\log T)^{1/2} \ll
    T^\varepsilon,
\end{multline*}
showing that 
\[
    \int_1^T \frac{\abs{\zeta((1-\delta) + \ic t)}}{\abs t} \dc t
    \ll (\log T) T^\varepsilon \ll T^\varepsilon \ll x^\varepsilon.
\]
Collecting errors, choosing $\alpha$ large, $c = 1+\varepsilon$ and $\delta =
1/2-\varepsilon$ reveals
\[
    \sum_{n \leq x} a(n) = Cx + O_{\varepsilon}(x^{1/2+\varepsilon}).
\]

\contactend

\end{document}
