\documentclass[a4paper,11pt]{article}
\pagenumbering{arabic}
\usepackage{../environment}

\author{Max von Consbruch}

\begin{document}

\begin{center}
    \huge{Solutions to Sheet 7.}
\end{center}

\section*{Problem 1}
\begin{enumerate}
\item[a)] We have $g(x) \ll x^{-(-u+av)}$ as $x \to 0$ and 
        $g(x) \ll x^{-(-u+bv)}$ as $x \to \infty$. Hence in $-u+av < 
        \Re(s)< -u+bv$ the mellin transform $\hat g$ exists and is given by
    \[
        \hat g (s) = \int_0^\infty x^uf(x^v)x^s \frac{\dc x}x 
        = v^{-1} \int_0^\infty f(y)y^{(s+u)/v-1} \dc y = v^{-1} f\left(
            \frac{s+u}v \right)
    \]
    Now the RHS defines a holomorphic function in $-u+a'v < \Re(s) < 
    -u + b'v$. 
\item[b)]Of course, knowing bounds for $f$ does not imply any bounds for 
    $f'$. But knowing that we can derive $f$, we can make use of partial
    integration. We have
    \[
        \int_0^\infty f(x) x^{s-1} \dc x = \left[f(x)
        \frac{x^s}s\right]_0^\infty - \frac 1s \int_0^\infty f'(x)x^s \dc x
    \]
    By assumption, the boundary terms vanish for $a < \Re (s) < b$, and
    the integral on the RHS exists (if this is not clear, try to first 
    approximate the integrals by truncated ones from $1/T$ to $T$ and let 
    $T \to \infty$). Hence  $\hat{g}$ (with $g=f'$) exists in $a+1 < \Re(s) <
    b+1$
    (note the shift $s \mapsto s+1$ in the integral). Same argument as
    before gives continuation of $\hat g$ to $a'+1 < \Re s < b'+1$. 
\item[c)] By assumption $f$ has compact support, so the Mellin Transform
    exists everywhere and the same holds for the derivatives. We make use
    of what we showed in $b)$ repeatedly, obtaining
    \[
        \hat f(s) = \frac{(-1)^N}{s(s+1)\dots(s+N-1)} \hat g(s+N)
        = (-1)^N \frac{\Gamma(s)}{\Gamma(s+N)} \hat g(s+N)
    \]
    where $g= f^{(N)}$. The first $\Gamma$-factor behaves (for 
    fixed real part and large imaginary part of $s$) like $O(\abs s^{-N})$, 
    so it remains to show that $\hat g(s)$ is bounded with $\Im s \to \infty$. But the integral
    from the mellin transform can be bounded in absolute values, as
    \[
        \abs{g(s)} \leq \int_0^\infty \abs{g(x) x^{s-1}} \dc x
        \ll \int \abs {g(x)} x^{\Re(s)-1} \dc x.
    \]
    This is convergent, and independent of $\Im(s)$. 
\item [d)] Calculation: 
    \begin{multline*}
        \hat{f \star h}(s) = \int_0^\infty (f \star h)(x) x^{s-1} \dc x
        = \int_0^\infty \int_0^\infty f(t) h(x/t) t^{-1} \dc t x^{s-1} \dc x
        \\ = \int_0^\infty f(t) h(y) t^{s-1}y^{s-1} \dc t \dc y,
    \end{multline*}
    as desired. We made use of the substitution $y = x/t$, i.e. $\dc y = 
    t^{-1} \dc x$. 
\end{enumerate}
\section*{Problem 2\&3}
\textbf{a)} We want to apply Perron. Remember that we showed earlier that
the Dirichlet series attached to the characteristic function on the set
of squarefull numbers is given by $\frac{\zeta(2s)\zeta(3s)}{\zeta(6s)}$. 
Just as in one of the examples from the lecture, we apply Perron with
$c = 1 + 1/\log x$ and $T = x^\alpha$ for some fixed $\alpha \in (0,1)$.
The absolute value of the coefficients is $\leq 1$ and we obtain
\[
    \sum_{n\leq x \  \rm sqfull} 1 
    = \pifrac \int_{c-\ic T}^{c+\ic T} \frac{\zeta(2s)\zeta(3s)}{\zeta(6s)}
    x^s \frac {\dc s}s + O(T^{-1} x \log x).
\]
We want to shift the contour to the left and pick up residues along the way. 
The most important tool to bound the vertical contribution is the moment bound,
and this requires the real part of the argument to be at least $\frac 12$. 
Hence we shift to $\Re s = \frac 14$. The factor $\zeta^{-1}(6s)$ is still
holomorphic here, so we only pick up the residues from $\zeta(2s)$ and $\zeta(3s)$.
We obtain
\begin{multline*}
    \sum_{n \leq x \ \rm sqfull} 1  =  \\
    \frac{\zeta(3/2)}{\zeta(3)}x^{1/2} + \frac{\zeta(2/3)}{\zeta(2)} x^{1/3}
    + \left(\int_{c-\ic T}^{1/4 - \ic T} + \int_{1/4 - \ic T}^{1/4 + \ic T} 
    + \int_{1/4 + \ic T}^{c + \ic T}\right) \frac{\zeta(2s)\zeta(3s)}{\zeta(6s)}
    x^s \frac{\dc s}{s} + O(T^{-1} x \log x).
\end{multline*}
First, note that $\zeta^{-1}(s)$ is bounded in $\Re s > 1+\delta$, as 
\[
    \abs{\zeta^{-1}(s) } = \prod_p \abs{1-p^{-s}} 
    \leq \prod_p (1 + p^{-1-\delta}) = \frac{\zeta(2+2\delta)}{\zeta(1+\delta)}
    \ll_\delta 1. 
\] 
So we disregard this factor from now on. Let us first start with the vertical part.
Here we have $\abs {x^s} = x^{1/4}$, so the contribution is bounded by 
\[
    \ll x^{1/4} \int_0^T \frac{\abs \zeta(1/2 + 2 \ic t)  \zeta(3/4 + 3 \ic t)}
    {1/4 + \ic t} \dc t.
\]
We prove that the integral is bounded by $x^{\varepsilon}$. By splitting the integral
into $\log x$ dyadic pieces $[M, 2M]$, is suffices to show that 
\[
    \int_M^{2M} \frac{\abs \zeta(1/2 + 2 \ic t)  \zeta(3/4 + 3 \ic t)} {1/4 +
    \ic t} \dc t.
\]
The numerator is $\geq M$, so we really only need to show that 
\[
    \int_M^{2M} \abs {\zeta(1/2 + 2 \ic t)  \zeta(3/4 + 3 \ic t)} \dc t \ll 
    M^{1+\varepsilon}.
\]
This is an immediate consequence of Cauchy-Schwartz and the moment bounds. 

Next, we focus on the horizontal parts. Here, $s^{-1} \ll T^{-1}$, so the
contribution become
\[
    \ll T^{-1} \int_{1/4}^c \abs{\zeta(2(\sigma+\ic T))\zeta(3(\sigma+\ic T)} \dc 
    \sigma \ll T^{-1} \int_{1/4}^c T^{\max(1/2-\sigma, 0)} T^{\max(1/2-3\sigma/2,0)}
    x^\sigma \dc \sigma.
\]
This requires some bookkeeping, but splitting this into the parts
$(1/4, 1/3)$, $(1/3, 1/2)$ and $(1/2, \sigma)$ one quickly verifies that 
no term contriibutes more that $x^{1+\varepsilon}$. To this end, we showed
\[
    \sum_{n \leq x \rm \ sqfull} 1 
    = \frac{\zeta(3/2)}{\zeta(3)}x^{1/2} + \frac{\zeta(2/3)}{\zeta(2)} x^{1/3}
    + O\left( \frac{x^{1+\varepsilon}}T + x^{1/4 + \varepsilon}\right).
\]
The claim follows upon setting $T = x^{3/4}$. 

\textbf{b)} The good thing with smooth weights is that their mellin transforms
usually decay quickly along vertical lines and we do not have to worry about cutting
off the integral. Perron's formula reveals with $c > 1/2$
\[
    \sum_{n \text{ squarefull}} \ec^{-n/x} = \pifrac \int_{(c)} \frac{\zeta(2s) \zeta(3s)}
    {\zeta(6s)} x^s \Gamma(s) \dc s.
\]
As $\Gamma$ vanishes rapidly along vertical lines, we can shift the contour to 
$\Re s = 1/6 + \varepsilon$ and obtain
\[
    \dots = \frac 12 \frac{\zeta(3/2)}{\zeta(3)} x^{1/2} + 
    \frac 13 \frac{\zeta(2/3)}{\zeta(2)} x^{1/3} + \pifrac \int_{(1/6 + \varepsilon)}
    \frac{\zeta(2s) \zeta(3s)} {\zeta(6s)} x^s \Gamma(s) \dc s.
\]
The integral is absolutely convergent, hence gives an error of size
$O(x^{1/6+\varepsilon})$.

Remark: We will later prove that $\zeta(s)$ does not have zeroes in some
neighbourhood of the line $\Re s = 1$, which in particular implies that there
are no zeroes on the line itself. Hence we can get even shift the contour onto
$\Re s = 1/6$, killing the $+\varepsilon$. 

\section*{Problem 4}
\textbf{a)} Every finite abelian group can be decomposed as a product of 
cyclic groups of prime-power-order. Hence the number of isomorphism classes 
of abelian groups of order $n$ gives a multiplicative arithmetic function
$$a: \N \to \N, \quad n \mapsto \#(\{\text{abelian groups of order
$n$}\}/\cong).$$ 
If $n = p^r$ is a prime power, we find that $a(n)$ is given by the number of 
(additive) partitions of $r$. Indeed, to a partition
\[
    1 \cdot a_1 + 2\cdot a_2 + 3 \cdot a_3 + \dots  = r 
\]
we can associate a group $(\Z/p\Z)^{a_1} \times (\Z/p^{2}\Z)^{a_2} \times
(\Z/p^3\Z)^{a_3} \times \dots$ of order $p^r$, and vice versa. One quickly verifies
(at least formally), that
\[
    \sum_{n=1}^\infty a(n) x^n = (1+x+x^2+\dots) (1+x^2+x^4+\dots) 
    (1+x^3+x^6+\dots)\cdots
\]
and substituting $x=p^{-s}$ for varying $p$ yields the desired formula
\[
    \sum_{n=1}^\infty a(n) n^{-s} = \prod_{p} \prod_{r=1}^\infty(1-p^{-rs})^{-1}
    = \prod_{r=1}^\infty \zeta(rs).
\]
The last step might demand clearification. Remember that a product $\prod a_n$
with $a_n \neq 0$ converges absolutely to something $\neq 0$ iff the sum $\sum
\abs{a_n-1}$ converges absolutely. In $\Re s > 1+\delta$ we have the uniform
bound
\[
    \abs{1-\zeta(s)} \ll \sum_{n=2}^\infty n^{-1-\delta} \ll_\delta 2^{-\delta},
\]
which shows that indeed, the product converges absolutely and locally uniformly in
$\Re s > 1$. 

\textbf{b)} The heuristic goes as follows. Let $F$ be the Dirichlet series attached
to $a$. By the above, $F$ is a holomorphic function for $s > 1$, but by 
the continuation of the first $\zeta$-factor, we find that $F$ has a
continuation to a meromorphic function on $\Re s > 1/2$. (Aside: We can apply
the functional equation to as many $\zeta$-factors as we want, yielding
continuations to $\Re s > 1/n$ for arbitrarily large $n \in \N$. But $F$ can
never be meromorphically continued to all of $\C$. This is because there are
poles at $s=1, 1/2, 1/3, \dots$, which by the identity theorem implies that
$F^{-1}=0$.) Now Perron's Formula reads
\[
    \sum_{n \leq x} a(n) = \pifrac \int_{(c)} F(s) x^s \frac{\dc s}s,
\]
and upon shifting the contour to $1-\varepsilon$ we obtain
\[
    \sum_{n \leq x}a(n) = x\text{Res}_{s=1} F(s) + \pifrac \int_{(1-\varepsilon)}
    F(s)x^s \frac{\dc s}s.
\]
The residue is given by $C = \zeta(2) \zeta(3) \cdots$, and we'd hope that we would be able to approximate the integral by something of size 
$o(x)$.

\textbf{Proving the asymptotic using Cesàro-weights.} Proving the asymptotic is
quite challenging, as we would have to find some bound on $a(n)$ to apply (4.7).
The convergence of $\sum_n a(n)n^{-s}$ for $\Re (s)>1$ gives $a(n) \ll n$, but 
there is no trivial way to get anything beyond that.
Further below is a solution assmuing $a(n) \ll n^{1/2+\varepsilon}$, but 
it turns out we don't need such bounds! Instead of bounding 
\[
    S_0(x) = \sum_{n \leq x} a(n),
\]
we choose to bound
\[
    S_1(x) = \sum_{n \leq x} a(n) (x-n) = \int_1^x S_0(y) \dc y.
\]
(These weights are called Cesàro weights). 
Integrating Perron's formula, we find that 
\[
    S_1(x) = \pifrac \int_{(c)} F(s) x^{s+1} \frac{\Gamma(s)}{\Gamma(s+2)} \dc s.
\]
The $\Gamma$-factor is essentially bounded by $\abs s^{-2}$, at least for 
$\abs s > 2$. Whenever $\sigma > 1/2 + \delta$ and $\abs t > 1$ we find  
$$F(\sigma + \ic t) \ll \abs{\zeta(s)} \zeta(1+2\delta) \zeta(3/2+3/2\delta) \cdots
\ll \abs t^{\frac {1-\sigma}2 + \varepsilon} \delta^{-1}.$$
Hence we can shift the contour to $\Re s = 1/2+\delta$, pick up a pole and the 
remaining integral remains absolutely convergent. In formulas,
\[
    S_1(x) = \frac{x^2}2 C + \int_{(1/2+\delta)} F(s) x^{s+1}
    \frac{\Gamma(s)}{\Gamma(s+2)} \dc s = \frac{x^2}2 C +
    O_\delta(x^{3/2+\delta}).
\]
Nice, this at least shows that there is an asymptotic \textit{on average}. But
how can we make use of this? We also showed that the Lindelöf-Hypothesis is true
\textit{on average}, but we are far from proving the Lindelöf-Hypothesis in
general! What plays in our favor here is that $S_0$ is non-decreasing. Denote
by $E_0(x)$ the error function $S_0(x) - Cx$, and define $E_1$ as the integral
of $E_0$. Note that we have $E_1(x) \ll x^{3/2+\varepsilon}$ by the above. 
We also make a choice of some $Q = x^\alpha$ for $\alpha \in [0,1]$ and get
(using monotonicity of $S_0$)
\[
    E_1(x+Q)-E_1(x) = \int_x^{x+Q} E_0(t) \dc t \geq Q(S_0(x) - Cx - CQ) = QE_0(x) + O(Q^2).
\]
But we also know that $E_1(x+Q)-E_1(x) = O(x^{3/2+\varepsilon})$, implying
\[
    QE_0(x) \leq O(x^{3/2+\varepsilon}+Q^2).
\]
This shows $E_0(x) \leq O(x^{3/4+\varepsilon})$ once we choose $Q=x^{3/4}$. A similar
lower bound can be established by inspecting $\int_{x-Q}^x E_0(t) \dc t$ (exercise, 
haha). This proves $S_0(x) = Cx + O(x^{3/4+\varepsilon})$. This really is remarkable,
as this in particular implies that $a(n) \ll n^{3/4 + \varepsilon}$, which is a 
bound we did not know existed beforehand. Even more, this followed only from a
bound on the vertical growth of $F(s)$ and the fact 
that $a(n) \geq 0$. The Lindelöf Hypothesis would imply an error term of 
size $O(x^{1/2+\varepsilon})$. 


\textbf{Proving the asymptotic assuming $a(n) \ll n^{1/2+\varepsilon}$. (I did
this first but then realized that we can avoid this.)} To prove
    the asymptotic, we first need to get some bound on $a(n)$. Using 
the asymptotic for the number of partitions, one should be able to show that 
$a(n) \ll n^{1/2+\varepsilon}$. (Might think about this again later.) 
Using (4.7), we find for $T = x^\alpha$ (to be chosen later) 
\[
    \sum_{n \leq x} a(n) = \pifrac \int_{(c)}^T F(s) x^s \frac{\dc s}s 
    + O(E),
\]
where $\int_{(c)}^T$ means $\int_{c-\ic T}^{c+\ic T}$ and 
\[
    E = x^{c-\alpha} \sum_n a_n n^{-c} + x^{1/2+\varepsilon}(1+ x^{1-\alpha}\log x)
    \ll_{c} x^{c-\alpha} + x^{1/2+\varepsilon} + x^{3/2 - \alpha + \varepsilon} \log x
\]
For $0< \delta < 1/2$ and $1-\delta < \Re s$ we have a uniform bound $F(s) \ll
(\delta-1/2)^{-1} \zeta(s)$ as the remaining factors are bound by $\zeta(2(1-\delta))
\zeta(3(1-\delta)) \cdots \ll (1/2-\delta)^{-1}$. Hence the convexity bound gives
$$F(\sigma + \ic t) \ll \abs t^{\frac {1-\sigma}2 + \varepsilon}$$
for $\abs t > 1$.
We are now ready for shifting the contour to $\Re s = 1-\delta$. Similarly to above,
we obtain
$$\sum_{n \leq x} = Cx + H(\delta) + \pifrac \int_{(1-\delta)}^T F(s) x^s
\frac{\dc s}s + O(E),$$
where the horizontal integrals are given and bounded by 
\[
    H(\delta) = \pifrac \left( \int_{c-\ic T}^{1-\delta - \ic T} - \int_{c+\ic
    T}^{1-\delta + \ic T} \right) F(s) x^s \frac{\dc s }s \ll (1/2-\delta)^{-1}
    x^c T^{\delta/2-1 + \varepsilon}.
\]
Only the vertical integral remains mysterios. We have $F(s) \ll (\delta-1/2)^{-1} \zeta(s)$, and obtain
\[
    V(\delta) = \pifrac \int_{(1-\delta)^T} F(s) x^s \frac{\dc s }s 
\ll x^{1-\delta}((1/2-\delta)\delta)^{-1} + (1/2-\delta)^{-1}x^{1-\delta}\int_1^T \frac{\abs{\zeta((1-\delta) + \ic t)}}{\abs t} \dc t
\]
We cut the integral in diadic pieces, which we can bound using 
the moment bounds, as
\begin{multline*}
    \int_T^{2T} \frac{\abs{\zeta(\sigma + \ic t)}}{\abs t} \dc t
    \ll \left(\int_T^{2T}\abs{\zeta(\sigma+\ic t)}^2 t^{-1} \dc t \right)^{1/2}
    \left( \int_T^{2T} t^{-1} \dc t\right)^{1/2} \\
    \ll (T^{-1} T^{1+\varepsilon})^{1/2} (\log T)^{1/2} \ll
    T^\varepsilon,
\end{multline*}
showing that 
\[
    \int_1^T \frac{\abs{\zeta((1-\delta) + \ic t)}}{\abs t} \dc t
    \ll (\log T) T^\varepsilon \ll T^\varepsilon \ll x^\varepsilon.
\]
Collecting errors, choosing $\alpha$ large, $c = 1+\varepsilon$ and $\delta =
1/2-\varepsilon$ reveals
\[
    \sum_{n \leq x} a(n) = Cx + O_{\varepsilon}(x^{1/2+\varepsilon}).
\]

\contactend

\end{document}
