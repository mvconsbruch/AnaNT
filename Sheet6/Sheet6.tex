\documentclass[a4paper,11pt]{article}
\pagenumbering{arabic}
\usepackage{../environment}

\author{Max von Consbruch}

\begin{document}

\begin{center}
    \huge{Solutions to Sheet 5.}
\end{center}

\section*{Problem 1}
Okay, we just go through everyting.
For $\zeta(s)$ we have degree $d=1$, conductor $N=1$, root number 
$\eta = 1$, $\kappa_1 = 0$ and hence $L_\infty(s) = \pi^{-s/2}\Gamma(s/2)$. 
For $L(s, \chi)$ with a primitive Dirichlet character $\chi$ mod $q>1$ we have
degree $d=1$, conductor $N=q$, root number $\eta = \ic^{-\kappa} \tau(\chi)q^{-1/2}$,
$\kappa_1 = \kappa$ and $L_\infty(s) = \pi^{-s/2}\Gamma(\frac{s+\kappa}2)$
where $\kappa = 0$ if $\chi$ is even and $\kappa = 1$ if $\chi$ is odd. 

The functional equation now reads as follows.
\begin{thm}[Approximate functional equation for $\zeta$]
    Let $G(u)$ be any even function which is holomorphic and bounded in 
    $\abs{\Re(u)} < 4$ and normalized by $G(0) = 1$. Let $X>0$. Then for 
    $0 < \sigma < 1$ we have 
    \[
        \zeta(s) =
        \sum_n n^{-s} V_s \left(\frac nX \right) + 
        \pi^{s-1/2} \frac{\Gamma((1-s)/2)}{\Gamma(s/2)} \sum_n 
        n^{s-1} V_{1-s}(nX) - R
    \]
    where
    \[
        V_s(y) = \pifrac \int_{(3)} G(u)  \frac{\Gamma((s+u)/2)}
        {\Gamma(s/2)}(y \sqrt \pi)^{-u} \frac{\dc u}u
    \]
    and
    \[
        R = \frac{\pi^{s/2}}{\Gamma(s/2)} \frac{G(1-s)}{1-s} X^{1-s}
        - \frac{\pi^{s/2}}{\Gamma(s/2)} \frac{G(-s)}{-s} X^{-s}.
    \]
\end{thm}
Completed Dirichlet $L$-functions are entire, so we get rid of $R$. 
As above, in the following $\kappa$ depends on the parity of $\chi$. 
\begin{thm}[Approximate functional equation for Dirichlet $L$-functions]
Let $G(u)$ be any even function which is holomorphic and bounded in 
    $\abs{\Re(u)} < 4$ and normalized by $G(0) = 1$. Let $X>0$. Then for 
    $0 < \sigma < 1$ we have 
    \[
        \zeta(s) =
        \sum_n \chi(n)n^{-s} V_s \left(\frac n{X\sqrt q} \right) + 
        \epsilon(s) \sum_n \overline{\chi(n)}
        n^{s-1} V_{1-s}\left(\frac{nX}{\sqrt q}\right)
    \]
    where
    \[
        V_s(y) = \pifrac \int_{(3)} G(u) 
        \frac{\Gamma((s+u+\kappa)/2)}
        {\Gamma((s+\kappa)/2)} (y \sqrt \pi)^{-u} \frac{\dc u}u
    \]
    and
    \[
        \epsilon(s) = \ic^{-\kappa} \tau(\chi) q^{-s} \pi^{s-1/2} \frac{\Gamma
        ((1-s+\kappa)/2)} {\Gamma((s+\kappa)2)}.
    \]
\end{thm}

As for (3.11), we have for $\zeta$ that $\mathcal C(s) = \CC_0(s) = \abs {s+2}$, 
for Dirichlet $L$-functions we find $\cC_0(s) = \abs{s + \kappa} + 2$ and
$\cC(s) = q(\abs{s + \kappa} + 2)$. As an aside, the $2$ here is quite arbitrary 
and is only there to make sure everything works out when $\abs s$ is small. 
We can plug this into (3.11), finding (with $G(u) = \ec^{u^2}$) that for $\zeta$
we have that
\[
    y^aV_s^{(a)}(y) \ll_{a,A} \left(1 + \frac{y}{\sqrt{\abs s +2}} \right)^{-A}
\]
for $\Re(s) > 0$ whereas for $L(s, \chi)$ we find
\[
    y^aV_s^{(a)}(y) \ll_{a,A} \left(1 + \frac{y}{\sqrt{\abs {s+\kappa} +2}}
    \right)^{-A}
\]
for $\Re(s) > -\kappa$. 

Lastly, the conditions for (3.12) are satisfied for both $\zeta(s)$ and $L(s,\chi)$, 
we have the convexity bound
\[
    \zeta(s) \ll_{\varepsilon, \delta} (\abs s + 2)^{\frac{1-\sigma}2+\varepsilon}
\]
whenever $\abs{s-1} \geq \delta$ (i.e., away from the pole) and similarly
\[
    L(s,\chi) \ll_\varepsilon (q\abs{s+\kappa}+2)^{\frac{1-\sigma}2 + \varepsilon}. 
\]
Again, it should be noted that the $2$ is added artificially to have 
small $\abs s$ not mess everything up. For large $s$, these vanish and we obtain 
(and should really read these as)
\[
    \zeta(s) \ll \abs s ^{\frac{1-\sigma}2+\varepsilon} \quad \text{and}
    \quad  L(s, \chi) \ll \abs {qs}^{\frac{1-\sigma}2 + \varepsilon}. 
\]
Also, if we fix $L$, we can absorb the factor $q$ into the implicit constant 
from $\ll$. 



\section*{Problem 2}
We first calculate $\zeta(0)$. The simple pole of $\zeta(s)$ at $s=1$ has
residue $1$, so we know that 
$\lim_{s \to 1} (s-1) \zeta(s) = 1$. Writing the functional equation as
$\zeta(s) = \Delta(s) \zeta(1-s)$ 
gives 
\[
    1 = \lim_{s \to 1} (s-1) \zeta(s) = \lim_{s \to 1} (s-1) \Delta(s) \zeta(0),
\]
so we only need to evaluate the remaining term $\lim_{s \to 1}(s-1) \Delta(s)$.
We have 
$$\Delta(s) = \frac{\Gamma(\frac{1-s}2)}{\Gamma(\frac s2)} \pi^{s-1/2}.$$
It follows that $\zeta(0) = -\frac 12$ as $\Gamma(1/2) = \sqrt \pi$ and 
$\Gamma((1-s)/2)$ has residue $-2$ at $1$ (think about the Laurent expansion at $1$
and remember that $\Gamma$ has residue $1$ at $0$). 

\textit{$\zeta(s) < 0$ for $s \in (0,1)$.} We have that 
\[
    \zeta(s) = \frac{s}{s-1} - s \int_0^\infty \{t\} t^{-s-1} \dc t.
\]
This is negative. Hence $\zeta$ is negative in the interval $[0,1)$. 

\section*{Problem 3}
We want to follow the proof from (3.14) as closely as possible. The first 
difference is that we sum over all characters, not just the primitive ones, 
but this does not make a difference: If we know that 
\[
    \ssum_{\chi \text{(mod $q$)}} \abs{L(1/2, \chi)}^2 \ll q^{1+\varepsilon}
\]
(where the star in the sum means that we sum over primitive characters, this 
notation is quite common), we can easily deduce
\[
    \sum_{\chi \text{(mod $q$)}} \abs{L(1/2, \chi)}^2 
    = \sum_{d \mid q}\ssum_{\chi \text{(mod $d$)}} \abs{L(1/2, \chi)}^2 
    \leq \tau(q) \max_{d \mid q} \ssum_{\chi \text{(mod $d$)}} \abs{L(1/2, \chi)}^2 
    \ll q^{1+\varepsilon}
\]
as $\tau(q)$ the number of divisors of $q$, satisfies $\tau(q) \ll q^{\varepsilon}$. 
The $L_\infty$ factor occuring in $V_s$ only depends on the parity of $\chi$, 
so we further split the sum into odd and even parts. We want to use the 
approximate functional equation with $X=1$, $s=1/2$ and $G(u) = \ec^{u^2}$ as in 
(3.11). Let's check what happens. We find
\[
    L(1/2, \chi) = \sum_n \frac{\chi(n)}{n^{1/2}} V_{1/2}(n/\sqrt N)
    +  \epsilon({1/2})\sum_n \frac{\overline\chi(n)}{n^{1/2}} V_{1/2}(n/\sqrt N) + R
\]
where 
\begin{itemize}
    \item $R = 0$ as the completed $L$-function $\Lambda(s, \chi)$ is entire.
    \item The root number $\epsilon(1/2)$ has absolute value $1$. 
    \item The terms involving $V = V_{1/2}$ can be bounded by 
        $V(y) \ll_A (1 + y)^{-A}$. For all $A > 0$. 
\end{itemize}
Also note that both sums are equal in absolute value. This is not too
complicated! We plug it in, using this time that $\abs{a+b}^4 \leq 8 (\abs a^4 
+ \abs b^4)$ (this can be seen using Hölder's inequality for example),
obtaining 
\[
    \ssum_{\chi \text{(mod q) even}} \abs{L(1/2, \chi)}^4 
    \leq 16 \ssum_{\chi(q) \text{ even}} \abs{\sum_n
    \frac{\chi(n)}{n^{1/2}}V(n/\sqrt q)}^4.
\]
Similar to the proof of (3.14), we can complete the sum to go over all characters 
and open up the sum, obtaining a fourfold sum which we can simplify using
orthogonality relations on sums over characters. 
In short, we get
\begin{equation}
    \dots 
    \leq 16 \sum_{\chi (q)} \abs{ \sum_n \frac{\chi(n)}{n^{1/2}}V(n/\sqrt q)}^4
    = 16 \sum_{n_1, n_2, m_1, m_2} \frac{V_{n_1} V_{n_2} \bar V_{m_1} \bar
V_{m_2}}{(n_1n_2m_1m_2)^{1/2}}
    \sum_{\chi (q)} \chi(n_1+n_2-m_1-m_2),
\end{equation}
where we wrote $V_n = V(n/\sqrt q)$.
The sum over $\chi$ does not vanish iff $n_1n_2 \equiv m_1m_2 \pmod q$, 
where it equals $\phi(q)$. As the hint suggests, we glue together $n_1$ and $n_2$,
$m_1$ and $m_2$, which leaves us with the taks of bounding terms of the form
\[
    (V * V)(n) = \sum_{n_1 n_2 = n} V_{n_1} V_{n_2}.
\]
We find for any $A \geq 1$
\[
    (V * V)(n) \ll \sum_{n_1n_2 = n} \left(1+\frac {n_1} {\sqrt q} \right)^{-A}
    \left(1+\frac {n_2} {\sqrt q} \right)^{-A} \leq 
    \sum_{n_1 n_2 = n} \left(1+\frac {n} {q} \right)^{-A}
    \ll_{\varepsilon}n^{\varepsilon} \left(1+\frac {n} {q} \right)^{-A}.
\]
With $A = 1+\varepsilon$ we calculate
\begin{equation}
    (1) \ll \phi(q) \sum_{n,m} \frac{(V\star V)(n) \overline{(V\star V)(m)}}{(nm)^{1/2}}
    \ll \phi(q) \sum_n \sum_{n \equiv m \geq n} 
    (1+\tfrac mq)^{-1} (1+\tfrac nq)^{-1} (mn)^{-1/2},
\end{equation}
and upon applying the bound $\phi(q) < q$ this is exactly the sum that
arises in the end of the proof of (3.14)! (I might add lines on how to bound
this once I have time).
 


\end{document}
