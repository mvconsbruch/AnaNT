\documentclass[a4paper,11pt]{article}
\pagenumbering{arabic}
\usepackage{../environment}

\author{Max von Consbruch}

\begin{document}

\begin{center}
    \huge{Solutions to Sheet 10.}
\end{center}

\textbf{Reminder:} $\Li(n) \coloneqq \int_2^n \frac 1{\log t} \dc t$. 

\section*{Problem 1}
One might think that this problem is incredibly complicated, but in reality it 
is terribly simple. Let $P = P(q,a)$ be the smallest prime congruent to $a$ mod $q$.
The idea is now to plug this into to use the that 
\[
    \psi_0(P(q,a)-1,q,a) = \sum_{p \equiv a \pmod q} \log p = 0,
\]
Siegel-Walfisz says that $\psi(x,q,a) \approx \psi_0(x,q,a) \approx \frac
x{\phi(q)}$ if $x$ is large, so we'd expect to find that $P(q,a)$ cannot be too
large if the equality above holds.

Okay. If 

\section*{Problem 2}
\section*{Problem 3}
Before solving this, we should maybe try to figure out why we would expect this result. 
Given some number $n$, we are supposed to evaluate the counting function
\[
    R(n) = \# \{p \leq n \mid \text{$n-p$ is square free}\}.
\]
Naively, one might be think that 
$$R(n) \approx \zeta(2)^{-1}\pi(n) = \prod_{p} (1 - p^{-2}) \pi(n),$$
as the propability of a random number to be square-free is (in a suitable sense) given by
$\zeta(2)^{-1}$, and we inspect numbers (which seem random) in a set of cardinality 
$\pi(n)$. This heuristic is not too far off, but it is wrong! The main term of the 
asymptotic is clearly different. \\ 
To see what goes wrong, let $q$ be any prime
number.  First assume that $q \nmid n$. What is the probability that $q^2$ divides $n-p$
for some prime $p\neq q$?  Neither $n$ nor $p$ are divisible by $q$, so the
residue classes of these
numbers mod $q^2$ are invertible, and there are $\phi(q^2)$ such residue
classes.  So the probability is given by $\phi(q^2)^{-1}$.  Now assume $q \mid
n$. One quickly checks that $q^2$ cannot divide $n-p$
(unless $p = q$, but this case does not contribute much). Now we can explain
the asymptotic: There are
$\approx \Li(n)$ primes $\leq n$, and the probability for $n-p$ not being divisible by 
some prime $q$ is given by $(1-\phi(q^2)^{-1})$ if $q \nmid n$ and by $1$ if $q \mid n$. 
As $n-p$ is square-free iff no square of a prime divides it, we should expect
\[
    R(n) \approx \prod_{q \nmid n}(1-\phi(q^2)^{-1}) \Li(n) = \prod_{q
    \nmid n}\left(1-\frac1{q(q-1)}\right)^{-1} \Li(n),
\]
and this is what we have to prove.

\textit{Proof.} Clearly, we have $R(n) = \sum_{p \leq n} \mu^2(n-p)$. A standard 
trick to deal with $\mu^2$ is writing it as $\mu(k) = \sum_{d^2 \mid k} \mu(d)$.
Applying this gives
\[
    R(n) = \sum_{p \leq n} \mu^2(n-p) = \sum_{p \leq n} \sum_{d^2 \mid n-p} \mu(d)
    = \sum_{d \leq \sqrt n} \mu(d) \sum_{p \leq n, \ p \equiv n \text{ mod } d^2} 1.
\]
This is now basically an issue of counting primes in an arithmetic progression! 
Hence it really smells like Siegel-Walfisz, but this is not applicable right away. 
One issue is that we can only apply Siegel-Walfisz if $(d,n)=1$. But restricting to those
$d$ does not really affect our main term, as whenever $(d,n)>1$ there is at
most one prime number in that arithmetic progression, and the contribution of
those is bounded by $\omega(n) \ll n^{\varepsilon}$. Furthermore, and more
seriously, Siegel-Walfisz is only applicable if $d$ is small compared to $n$,
more precisely, only if $d < (\log n)^A$.  But again, we can elementarily bound
the terms with $d > (\log n)^A$. Given some $d$, the amount of numbers $<n$
congruent to $n$ mod $d^2$ can be bounded by $\ll \frac n {d^2}$. We obtain
\[
    R(n) = \sum_{d \leq (\log n)^A, \ (d,n)=1} \psi(n; n, d^2) + 
    O\left(\sum_{(\log n)^A < d < \sqrt n} \frac{n}{d^2} \right) + O(\sqrt n),
\]
and the $O$-terms can be bound by $\ll \frac{n}{(\log n)^A}$. Also, we can now apply
Siegel-Walfisz! We find
\[
    R(n) = \sum_{d \leq (\log n)^A, \ (d,n)=1} \frac{1}{\phi(d^2)} \Li(n) 
    + O\left( \frac{n}{(\log n)^A} \right).
\]
The sum can be completed, as $\phi(d^2) \gg \frac {d^2}{\log \log d} \gg
d^{2-\varepsilon}$, so that 
\[
    \sum_{d > (\log n)^A} \frac 1{\phi(d^2)} \ll \frac{1}{(\log n)^{A(1-\varepsilon)}}.
\]
This allows us to conclude (for any $A$, not the choice we made before)
\[
    R(n) = \sum_{d \in \N, \ (d,n) = 1} \frac 1 {\phi(d^2)} \Li(n)
    + O_A\left( \frac{n}{(\log n)^A} \right) = \prod_{p \nmid n}\left(1-\frac
    1{\phi(p^2)}\right) \Li(n) + O_A\left( \frac{n}{(\log n)^A} \right).
\]




\section*{Problem 4}
\contactend

\end{document}
