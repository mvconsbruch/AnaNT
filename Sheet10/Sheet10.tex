\documentclass[a4paper,11pt]{article}
\pagenumbering{arabic}
\usepackage{../environment}

\author{Max von Consbruch}

\begin{document}

\begin{center}
    \huge{Solutions to Sheet 10.}
\end{center}

\section*{Problem 1}
One might think that this problem is incredibly complicated, but in reality it 
is terribly simple. Let $P = P(q,a)$ be the smallest prime congruent to $a$ mod $q$.
The idea is now to plug this into to use the that 
\[
    \psi_0(P(q,a)-1,q,a) = \sum_{p \equiv a \pmod q} \log p = 0,
\]
Siegel-Walfisz says that $\psi(x,q,a) \approx \psi_0(x,q,a) \approx \frac
x{\phi(q)}$ if $x$ is large, so we'd expect to find that $P(q,a)$ cannot be too
large if the equality above holds.

Okay. If 

\section*{Problem 2}
\section*{Problem 3}
\section*{Problem 4}
\contactend

\end{document}
