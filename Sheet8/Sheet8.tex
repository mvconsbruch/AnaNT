\documentclass[a4paper,11pt]{article}
\pagenumbering{arabic}
\usepackage{../environment}

\author{Max von Consbruch}

\begin{document}

\begin{center}
    \huge{Solutions to Sheet 8.}
\end{center}

\section*{Problem 1}
\textbf{a-4p)} Look at sheet 5.

\textbf{b-6p)} The approximation in (4.9) reads 
\[
    \sum_{n \leq x} d_k(n) = x P_k(\log x) + O(x^{1-\delta}).
\]
Reading the proof reveals that the main term is given by the residue
\[
    R \coloneqq \Res_{s=1} \frac{\zeta^k(s) x^s}s = \Res_{s=1} F_k(s),
\]
where for convenience $F_k(s) \coloneqq \frac{\zeta^k(s) x^s}s.$
Of course $\zeta^k(s)$ has a singularity of degree $k$ at $1$, and we only need 
to calculate the $(-1)$st term of the Laurent expansion of $F$ at $1$. 
We have the Taylor expansions
\[
    \frac 1{s} =  \sum_{n=0}^\infty (-1)^n (s-1)^n = 1 - (s-1) + (s-1)^2 + O((s-1)^3)
\]
and
\[
    x^s = \sum_{n=0}^\infty x(\log x)^n (s-1)^n = x + x(\log x)(s-1) + x(\log x)^2(s-1)^2 + O((s-1)^3). 
\]
Let $a_n$ denote the coefficients of the Laurent series of $\zeta$ at $1$, i.e.
\[
    \zeta(s) = \sum_{n = -1}^\infty a_n (s-1)^n.
\]
Calculating $P_2$ and $P_3$ now is pure calculation. 

\textbf{Calculating $P_2$.} We find
\[
    \zeta^2(s) = \left( \sum_{n=-1}^\infty a_n s^n \right)^2
    = a_{-1}^2 (s-1)^{-2} + 2a_{-1}a_0 (s-1)^{-1} + O(1).
\]
We multiply this with the Taylor series above and find that the coefficient of 
$(s-1)^{-1}$ is given by 
\[
    2a_{-1}a_0 x + a_{-1}^2 (x \log x - x) = x ( a_{-1}^2 \log x + 2a_{-1}a_0-a_{-1}^2).
\]
\textbf{Remark.} It is possible to show by elementary means that 
\[
    \sum_{n\leq x}^\infty d_2(n) = x \log x + (2\gamma-1)x + O(x^{1/2}),
\]
which shows that $a_0 = \gamma$. (I just realized that $a_{-1} = 1$ is already
known haha, but we could probably also derive this with a similar approach and
$k=1$).

\textbf{Calculating $P_3$.} We find similarly to above
\[
    \zeta^3(s) = (s-1)^{-3} + 3a_0 (s-1)^{-2} + 3(a_0^2 + a_1)(s-1)^{-1} + O(1).
\]
and again use this to figure out the coeffitient of $(s-1)^{-1}$ in the 
Laurent expansion of $F_3$ around $s=1$. We find that this coefficient is given by
\begin{multline*}
    3(a_0^2 + a_1)x + 3a_0x ( \log x - 1) + x ((\log x)^2 - \log x + 1) \\
    = x ( (\log x)^2 + (\log x)(3 \gamma - 1) + 3 (\gamma^2 + a_1 - \gamma) + 1)
\end{multline*}





\section*{Problem 2}
\textbf{a)}
To calculate $\zeta'(0)$, we make use of the functional equation, in the
form
\[
    \zeta(1-s) = \zeta(s) \cdot \frac{2\Gamma(s)}{(2\pi)^s} 
    \sin((\pi(1-s)/2).
\]
(This can be derived from the usual functional equation using the reflection formula
$\Gamma(z) \Gamma(1-z) = \frac \pi{\sin(\pi z)}$. For $s$ close to $1$ we find
\[
    (s-1)\zeta(s) \Gamma(s) = 1 + O((s-1)^2),
\]
and 
\[
    \sin(\pi(1-s)/2) = -\tfrac \pi 2 (s-1) + O((s-1)^3).
\]
Hence
\[
    \zeta(1-s) = - \frac{\pi}{(2\pi)^s} + O((s-1)^2) = 
    - \frac 12 - \frac 12 \log(2\pi) (s-1) + O((s-1)^2). 
\]
This gives $\zeta'(0) = - \frac{\log 2 \pi}2$.
We now insert this into (5.2). For $L(s) = \zeta(s)$, this reads as
\[
    - \frac{\zeta'(s)}{\zeta(s)} = - \frac{\log \pi}2 + \frac 12 \frac{\Gamma'(s/2)}
    {\Gamma(s/2)} - b + \frac 1s + \frac{1}{s-1} - \sum_{\rho \neq 0,1} 
    \left( \frac{1}{s-\rho} + \frac 1 \rho\right).
\]
Taking the logarithmic derivative of the recurrece relation $\Gamma(s+1) = s \Gamma(s)$ reveals
\[
    \frac{\Gamma'(s+1)}{\Gamma(s+1)} = \frac 1s + \frac{\Gamma'(s)}{\Gamma(s)}.
\]
We can use this to simplify our equation, leaving us with
\[
    -\frac{\zeta'(s)}{\zeta(s)} = -\frac{\log \pi}2 + \frac 12 \frac{\Gamma'(s/2+1)}{\Gamma(s/2 + 1)} - b + \frac {1}{s-1} - \sum_{\rho \neq 0,1} \left( \frac1{s-\rho} 
        + \frac 1 \rho \right ).
\]
When inserting $s=0$, this sum vanishes, and we find
\[
    -\frac{\zeta'(0)}{\zeta(0)} = -\frac{\log \pi}{2} + \frac 12 \frac{\Gamma'(1)}{\Gamma(1)} - b - 1.
\]
This proves the claim, as we know $\zeta'(0) = - \frac 12$, $\Gamma(1) = 1$ 
and the values for $\zeta$ and $\Gamma$ from above. 

\textbf{Remark.} The formula $\Gamma'(1) = -\gamma$ can be derived from
the Weierstraß product
\[
\frac 1 {\Gamma(z)} = z \ec^{\gamma z} \prod_{j=1}^\infty\left(1+ \frac zj
\right) \ec^{-z/j}
\]
by taking logarithmic derivatives on both sides and inserting $z=1$.

\textbf{b)} This statement is false, but of course, we are supposed to show 
$\abs{\Im(\rho) } \geq 6$ for all the \textit{non-trivial} roots of the zeta
function.
By (5.3a), we have that $$-b = -\Re b = \sum_\rho \Re(\tfrac 1 \rho).$$
The idea is that if $\Im \rho_1$ was small, then this sum would be so large that
this equality cannot hold (remember that $b = -0.023$ is quite small).
Let $\rho = \sigma + \ic t$ be a root with smallest possible imaginary value.
Note that with $\rho$, we also have roots $1-\sigma \pm \ic t$ and $\sigma - \ic t$,
so we may assume that $\sigma \geq \frac 12$ and that $t > 0$. As all
contributions in the sum of (5.3a) are positive, we find
\[
    -b \geq \frac 1{\sigma + \ic t} + \frac1{\sigma - \ic t} =
    \frac{2 \sigma}{\sigma^2 + t^2} \geq \frac 1{1+q^2}.
\]
This shows $t \geq \sqrt{-b^{-1}-1} \approx 6.5036$. 


\section*{Problem 3}
\textbf{a-5p)}
    The "only" thing left to made precise is the contour shift. The main ingrediants
    are (5.3b) and (5.3c).

    Just for convenience, let's summarize the bounds we need for
    $\tfrac{\zeta'}\zeta$:
    \[
        \frac{\zeta'}{\zeta}(s) = 
        \begin{cases}
            O(1) \quad &\Re s \geq 2, \\
            O(\log \abs s) \quad &\Re s \leq -\frac 12, \ \abs{s+2m} \geq \frac 14 \text{ for all $m \in \N$,} \\
            \sum_{\abs{\rho - s} \leq 1} \frac{1}{s-\rho} + O(1 + \log \abs s),
            \quad & -1 \leq \Re s \leq 3. 
        \end{cases}
    \]
    The first bound follows from $\frac{\zeta'}\zeta(s) = \sum_n \Lambda(n) n^{-s}$,
    the second part follows from the first bound, the functional equation and
    Stirling's formula. The third part is (5.3c).

    By (5.3b), there are approximately $\log T$ roots of the zeta-function
    with imaginary part close to $T$ (i.e., $\abs{T- \Im \rho} \leq 1$).
    Hence given some $n \in \Z$, the pigeonhole principle assures that it is 
    possible to find some $T = T_n$ with $\abs{n-T} \leq 1$ and 
    $\min_{\rho}{\abs{T - \Im\rho}} \leq \frac 1{\log \abs n}$. Together
    with $(5.3c)$, this gives that $\frac{\zeta'}{\zeta}(\sigma + \ic
    T) \ll (\log \abs T)^2$ on that line. 


    The plan is now to choose some large $n$, and 
    shifting the truncated integral 
    \begin{equation}
        -\pifrac \int_{2-\ic n}^{2+ \ic n} \frac{\zeta'}{\zeta}(s) \hat \omega(s)
        \dc s
    \end{equation}
    to $\Re s = -1/2$. This leaves us with the exercise to bound the horizontal
    integrals along the segments $[2\pm \ic n, -1/2 \pm \ic n]$. By the rapid decay
    of $\hat \omega$ and the bounds for $\zeta'/\zeta$, changing the boundaries
    of the integral in (1) from $[2-\ic n, 2 + \ic n]$ to $[2 - \ic T_n, 2 +
    \ic T_n]$ comes only with a small cost of $o(1)$. So we may also assume that
    uniformly $\zeta(\sigma + \ic T_n) \ll (\log n)^2$ along the horizontal segments. 
    This justifies the first contour shift, and we obtain
    \begin{multline*}
        \sum_n \Lambda(n) \omega(n) = \pifrac \int_{2- \ic n}^{2 + \ic n} 
        \frac{\zeta'(s)}{\zeta(s)} \hat \omega(s) \dc s + o(1) \\
        = \sum_{\abs{\Im \rho} \leq T_n} \hat \omega(\rho)
        + \pifrac \int_{-1/2 - \ic T_n}^{-1/2 + \ic T_n}
        \frac{\zeta'(s)}{\zeta(s)} \hat \omega(s) \dc s + o(1).
    \end{multline*}
    We may let $n \to \infty$, obtaining 
    \[
        \sum_n \Lambda(n) \omega(n) - \sum_\rho \hat \omega(\rho)
         = \int_{(-1/2)}  \frac{\zeta'(s)}{\zeta(s)} \hat \omega(s) \dc s.
    \]
    In the proof of (4.4c) we can abuse the fact that $\supp(\omega) \subset
    [2, \infty)$ to show that $\hat \omega(s) \ll \frac{2^{-s}}{\abs{\Im(s)}^N}$
    for all $N$. This shows the bound
    \[
        \int_{(-A+1/2)}  \frac{\zeta'(s)}{\zeta(s)} \hat \omega(s) \dc s
        \ll 2^{-A},
    \]
    and the shift is justified.

\textbf{b-5p)}
The observation is that this function has large negative peaks at the primes
(and smaller ones with sign $(-1)^k$ when $n=p^k$ is a prime power). Although 
the solution will (implicitely) assume the Riemann conjecture, this illustrates
the fact that \textit{$\zeta$ knows everything about the primes.}

Okay, let's analyze what's happening here. First, we have 
\[
    \cos(\gamma_j \log x) = \Re(\ec^{\ic \gamma_j \log x}) = \Re(x^{\ic \gamma_j}),
\]
so we are plotting the real part of the sum of $x^{\ic \gamma}$ over the first few
zeroes. If we want to interpret this as a sum $\sum_\rho \hat \omega(\rho)$, 
we would like to choose $\omega$ in way such that $\hat \omega(1/2 + \ic \gamma)
\approx x^{\ic \gamma}$ for the zeroes we want to consider, and $\hat \omega$ 
decaying rapidly after that range (ignoring the contribution of the trivial zeroes).

Apparently (not clear to me how to come up with this but hey it works) a
convenient seems to be 
\[
    \hat \omega(1/2 + \ic \gamma) = x^{\ic \gamma} \exp(-(\gamma/S)^2) = 
    x^{s-\frac 12} \exp\left( \left(\frac{s-1/2}{S}\right)^2\right),
\]
as this vanishes quickly once $\gamma > S$. We choose $S$ to be a parameter roughly
of the size of the largest zero we want to consider, which in our case is 
$\gamma_{30} \approx 100$. That's why we choose $S = 100$. We will later show
that the weight 
\[
    \omega(y) = \omega_{S,x}(y) = \frac {S}{2(\pi y)^{1/2}} \exp \left(
        -\left( \frac S2 \log\left(\frac yx\right) \right)^2
    \right)
\]
is the inverse Mellin-transform of $\hat \omega$. Now this is large if 
the part in the exponential vanishes, i.e., if $x \approx y$. On the other side,
if $y$ is not close to $x$ then $\log(y/x)$ becomes large (say of size
$\approx \frac{10}S$), then the factor $\exp(-S^2/4 \log(y/x)^2)$ makes
$\omega$ decay quickly. So at least for $x$ not too large (for large $x$ we need
a further distance between $y$ and $x$ to make $\log(y/x)$ become large), $\omega$
essentially looks like a peak at $y=x$. 
We are now ready to explain what's going on. With the explicit formula
(which we are technically not even allowed to use as $\omega$ is not compactly
supported, but whatever), we find
\[
    \sum_n \Lambda(n) \omega_{S,x}(n) \approx - \sum_{\abs{\Im \rho} \leq S} \hat
    \omega(\rho) \approx \sum_{j=1}^S \cos(\gamma_j \log x).
\]
If now $x \approx p^k$ is close to a prime power, the LHS is $\approx \Lambda(n)
\frac{S}{y^{1/2}}$, large. If not, there is no term on the LHS that
contributes much, so we would expect the RHS to be small. 

\textbf{Prove that "$\hat \omega = \hat \omega$".} We put $\hat \omega$ in the
inverse mellin transform to find
\[
    \omega(y) = \pifrac \int_{(c)} x^{s-1/2} \exp\left( \left(
    \frac{s-1/2}{S}\right)^2 \right) y^{-s} \dc s
\]
for all real numbers $c$. We substitute $u = \frac{s-1/2}S$ and find
\[
    \omega(y) = \frac S{y^{1/2}} \cdot \pifrac \int_{(c)}
    \exp(u^2) \left( \frac yx \right)^{-Su} \dc u.
\]
Abbreviating $v = S \log \frac yx$ shows further that 
\[
    \frac S{y^{1/2}} \pifrac \int_{(c)} \exp(u^2) \exp(-uv) \dc u 
    = \frac{S \exp(-v^2/4)}{y^{1/2}} \cdot \pifrac \int_{(c)}
    \exp((u-v/2)^2) \dc u.
\]
This integral does not depend on $c$, hence we may wlog assume $c=v/2$, which reveals
that this integral equals
\[
    \pifrac \int_{(0)} \exp(u^2) \dc u = \frac 1{2\pi} \int_{-\infty}^{\infty}
    \exp(-t^2) \dc t = \frac 1{2 \sqrt \pi},
\]
just what we wanted.



\section*{Problem 4}
First, an aside on the weird-looking error term $\psi(x)-x \ll x \ec^{-c
\sqrt{\log x}}$. On the one side it is better than every error term
of the form $x /(\log x)^A$ (for $A \in \R_{>0}$ large), on the other side 
it is worse than every error term of the form $x^{1-\delta}$ would be (for
$\delta \in \R_{>0}$ small).

Our version of the prime number theorem reads
\[
    \psi(x) = \sum_{p^n \leq x} \log p =  x + O(x\ec^{-c \sqrt{\log x}})
\]
for some constant $c > 0$. We deduce a formula for $\pi$ in two steps. First we show that 
$\psi(x)$ does not differ too much from the weighted prime-counting function
\[
    \psi_0(x) \coloneqq \sum_{p \leq x} \log p.
\]
Then we use $\psi_0$ for partial summation, utilizing that
\begin{equation}
    \pi(x) = \sum_{p \leq x} \frac{\log p}{\log p} = \frac {\psi_0(x)}{\log x}
    + \int_2^x \frac{\psi_0(t)}{t (\log t)^2} \dc t.
\end{equation}
Evaluating this should be possible using the approximation for $\psi_0(x)$. 

Let's carry this through, beginning with the estimate for $\abs{\psi(x)-\psi_0(x)}$. 
We find
\[
    \psi(x)-\psi_0(x) = \sum_{p^k \leq x, \ k \geq 2} \log p
    \leq \left(\sum_{p \leq \sqrt x} + \sum_{p \leq x^{1/3}} + \dots + 
    \right) \log x
\]
Note that there are at most $\log_2 x$ summation signs which don't run over an
empty set,
and every index set contains (trivially) less than $\sqrt x$ primes. We obtain
\[
    \psi(x) - \psi_0(x) \leq  (\log_2 x) \sqrt x (\log x) \ll x^{1/2+\varepsilon}.
\]
Now $\psi_0$ satisfies the same approximation as $\psi$, as
\[
    \psi_0(x) = \psi(x) + O(x^{1/2+\varepsilon}) = x + O(x \ec^{-c \sqrt {\log x}}).
\]
Inserting this in (1) yields
\[
    \pi(x) = \frac x{\log x} + \int_2^x \frac{1}{(\log t)^2} \dc t
    + O( x \ec^{-c \sqrt{\log x}}),
\]
where we used that $\int_2^x \frac{1}{t (\log t)^2} \dc t \ll 1$. As
\[
    \int_2^x \frac{1}{(\log t)^2} \dc t = \left[ \Li(t)-\frac t{\log t} \right]_2^x
    = \Li(x) - \frac{x}{\log x} + O(1),
\]
the claim follows. 



\contactend

\end{document}
