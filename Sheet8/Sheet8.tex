\documentclass[a4paper,11pt]{article}
\pagenumbering{arabic}
\usepackage{../environment}

\author{Max von Consbruch}

\begin{document}

\begin{center}
    \huge{Solutions to Sheet 8.}
\end{center}

\section*{Problem 1}
\textbf{a-4p)} I already gave a solution to this on sheet 5. 
\section*{Problem 2}
\section*{Problem 3}
\section*{Problem 4}
First, an aside on the weird-looking error term $\psi(x)-x \ll x \ec^{-c
\sqrt{\log x}}$. On the one side it is better than every error term
of the form $x /(\log x)^A$ (for $A \in \R_{>0}$ large), on the other side 
it is worse than every error term of the form $x^{1-\delta}$ would be (for
$\delta \in \R_{>0}$ small).

Our version of the prime number theorem reads
\[
    \psi(x) = \sum_{p^n \leq x} \log p =  x + O(x\ec^{-c \sqrt{\log x}})
\]
for some constant $c > 0$. We deduce a formula for $\pi$ in two steps. First we show that 
$\psi(x)$ does not differ too much from the weighted prime-counting function
\[
    \psi_0(x) \coloneqq \sum_{p \leq x} \log p.
\]
Then we use $\psi_0$ for partial summation, utilizing that
\begin{equation}
    \pi(x) = \sum_{p \leq x} \frac{\log p}{\log p} = \frac {\psi_0(x)}{\log x}
    + \int_2^x \frac{\psi_0(t)}{t (\log t)^2} \dc t.
\end{equation}
Evaluating this should be possible using the approximation for $\psi_0(x)$. 

Let's carry this through, beginning with the estimate for $\abs{\psi(x)-\psi_0(x)}$. 
We find
\[
    \psi(x)-\psi_0(x) = \sum_{p^k \leq x, \ k \geq 2} \log p
    \leq \left(\sum_{p \leq \sqrt x} + \sum_{p \leq x^{1/3}} + \dots + 
    \right) \log x
\]
Note that there are at most $\log_2 x$ summation signs which don't run over an
empty set,
and every index set contains (trivially) less than $\sqrt x$ primes. We obtain
\[
    \psi(x) - \psi_0(x) \leq  (\log_2 x) \sqrt x (\log x) \ll x^{1/2+\varepsilon}.
\]
Now $\psi_0$ satisfies the same approximation as $\psi$, as
\[
    \psi_0(x) = \psi(x) + O(x^{1/2+\varepsilon}) = x + O(x \ec^{-c \sqrt {\log x}}).
\]
Inserting this in (1) yields
\[
    \pi(x) = \frac x{\log x} + \int_2^x \frac{1}{(\log t)^2} \dc t
    + O( x \ec^{-c \sqrt{\log x}}),
\]
where we used that $\int_2^x \frac{1}{t (\log t)^2} \dc t \ll 1$. As
\[
    \int_2^x \frac{1}{(\log t)^2} \dc t = \left[ \Li(t)-\frac t{\log t} \right]_2^x
    = \Li(x) - \frac{x}{\log x} + O(1),
\]
the claim follows. 



\contactend

\end{document}
