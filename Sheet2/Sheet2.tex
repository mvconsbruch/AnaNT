\documentclass[a4paper,11pt]{article}
\pagenumbering{arabic}
\usepackage{../environment}

\begin{document}

\section*{Exercise 1}
\textbf{1.} The squarefull non-squares up to onehundred are $8,27,32,72$.

\textbf{2.} It suffices to show that any squarefull prime power can be written uniquely as
$p^k = a^2 b^3$ with $b$ square-free. But this is the same as writing $k =
2a+3b$ with $0 \leq b \leq 1$, and this is possible in a unique way once $k \geq 2$. 

\textbf{3.} Using the above and that $b$ is square-free iff $\mu^2(b) = 1$, we may write
\[
    \sum_{n \text{ squarefull}} n^{-s} = \sum_{a,b} \frac{\mu^2(b)}{a^{2s} b^{3s}} = \zeta(2s) \sum_{b} \mu^2(b) b^{-3s}.
\]
We can extend the Dirichlet series of $\mu^2$ into an Euler product, obtaining
\[
    \sum_{n \in \N} \mu^2(n)n^{-s} = \prod_p \left(1 + p^{-s} \right) = \prod_p
    \frac {\left(1 - p^{-s}\right)^{-1}} {\left(1 - p^{-2s}\right)^{-1}} =
    \frac {\zeta(s)}{\zeta(2s)}.
\]
(In the second-to-last equality we used $(1+x)(1-x) = 1-x^2$.)
We find 
\[
    \sum_{b} \mu^2(b) b^{-3s} = \frac{\zeta(3s) }{\zeta(6s)},
\]
done.

\section*{Exercise 2}
This is just a messy calculation. 
We somehow want to get of the $(a,b)$-symbol in the sum. We do so by using that
given $a,b \in \N$, we find unique coprime numbers $k, l$ with $a = k (a,b)$ and 
$b = k (a,b)$. Now summing over all possible gcds $d$ yields
\[
    \sum_{a,b} \frac{(a,b)}{a^s b^t} = \sum_{d} \frac d{d^{s+t}} \sum_{k,l
    \text{ coprime}} k^{-s}l^{-t} = \zeta(s+t-1)\sum_{k,l} k^{-s}l^{-t} \sum_{e \mid (k,l)} \mu(e)
\]
where we rephrased the coprimality condition on $k$ and $l$ using the trick
from the last sheet.
Now we rewrite
\[
    \sum_{k,l} k^{-s}l^{-t} \sum_{e \mid (k,l)} \mu(e) = \sum_e \mu(e)
    \sum_{k,l} (ke)^{-s} (le)^{-t} = \frac{\zeta(s)\zeta(t)}{\zeta(s+t)},
\]
obtaining
\[
    \sum_{a,b} \frac{(a,b)}{a^s b^t} = \frac {\zeta(s+t-1) \zeta(s) \zeta(t)}{\zeta(s+t)}.
\]
Tracing through this calculation, we find that it is sufficient for absolute convergence to have
$\Re(s)>1$ and $\Re(t) > 1$. These conditions are easily seen to be necessary too (the sub-sums with 
$a=1$ or $b=1$ diverge otherwise).

\textbf{Notes after correcting.} \leavevmode
\begin{itemize}
    \item Even though it is easily seen that the double sum cannot converge
        absolutely whenever (say) $\Re(s) \leq 1$, this does immediately follow from splitting 
        the sum! The reason is that it is possible to split a convergent sum into divergent 
        ones, i.e.
        \[
            \sum_{n \in \N} 0 = \sum_{n \in \N} (1-1) \neq \sum_n 1 - \sum_n 1.
        \]
\end{itemize}

\section*{Exercise 3}
\textbf{1.} We have
\[
    \psi(s) = \sum_{n} n^{-s} - 2 \sum_n (2n)^{-s}
\]
and 
\[
    \Tilde \psi(s) = \sum_{n} n^{-s} - 3 \sum_n (3n)^{-s}.
\]

\textbf{2.} Using the Leibniz criterion, we see that the series converge conditionally on the
positive real line, and thereby for $\Re s > 0$. 

\textbf{3.} As both $\psi$ and $\Tilde \psi$ are holomorphic in $\Re s > 0$, $\zeta$ can only
have a pole whenever $(1-2^{1-s})$ and $(1-3^{1-s})$ vanish. But this is the case whenever 
\[
    1 = 2^{1-s} = \ec^{(\log 2)(1-s)} \quad \Leftrightarrow \quad (\log 2)(1-s) \in 
    2 \pi \ic \Z
\]
and 
\[
    1 = 3^{1-s} = \ec^{(\log 3)(1-s)} \quad \Leftrightarrow \quad (\log 3)(1-s) \in 
    2 \pi \ic \Z.
\]

\textbf{4.} If $\log 2 / \log 3 = p/q$ was rational, we'd find that $2^p = 3^q$, contradiction.
Hence the two sets $\log 2 (2 \pi i \Z)$ and $\log 3 ( 2 \pi i \Z)$ have intersection the set
$\{0\}$. Thereby, $\zeta$ cannot have a pole away from $s = 1$. There it has a pole from a 
theorem in the lecture, and it is a simple pole as $(2^{1-s}-1)$ has a simple zero at $s=1$. 


\section*{Exercise 4}
We know that the $d$-th cyclotomic polynomial $\Phi_d(x)$ has degree $\phi(d)$, and that 
$\prod_{d \mid n} \Phi_d(x) = x^n - 1$. Hence 
\[
\sum_{d \mid n} \phi(d) = \sum_{d \mid n } \deg \Phi_d = \deg \left(\prod_{d
\mid n} \Phi_d \right) = \deg(x^n - 1) = n,
\]
hence (by Möbius-inversion)
\[
    \phi(n) = (\mu \star \id)(n) = \sum_{d\mid n}\frac nd \mu(d). 
\]
Now we find 
\[
    \sum_{n \leq x} \phi(n) / n = \sum_{n \leq x} \frac 1n \sum_{d \mid n}
    \frac nd \mu(d) = \sum_{d \leq x} \frac{\mu(d)}d \sum_{k : kd \leq x} 1 = 
    \sum_{d \leq x} \frac{\mu(d)}d \left[\frac xd \right] 
\]
We write $[x/d] = x/d + O(1)$ and use that $\mu(d) \in \{-1, 0, 1\}$. This gives 
\[
    \sum_{d \leq x} \frac{\mu(d)}d \left[\frac xd \right] 
    = \sum_{d \leq x} \frac{\mu(d)}d \frac xd + O\left(\sum_{d \leq x} \frac 1d \right) 
    = \sum_{d \leq x} \frac{\mu(d)}d \frac xd + O(\log x)
\]
(by approximating the $n$-th harmonic number with the logarithm) and 
we have 
\[
    \sum_{d \leq x} \frac{\mu(d)}d \frac xd = x \sum_{d = 1}^\infty \mu(d) d^{-2} + 
    O \left(x \sum_{x < d < \infty} d^{-2} \right) = x \zeta(2)^{-1} + O(1).
\]
One can show the estimate $\sum_{x < d < \infty} d^{-2} \ll x^{-1}$ using the inequality
\[
    \sum_{x < d < \infty} d^{-2} \leq \int_{x-1}^\infty t^{-2} \dc t = O((x-1)^{-1}) = O(x^{-1}).
\]
Done. 

\textbf{Notes after correcting.} \leavevmode
\begin{itemize}
    \item The convolution formula can also be obtained formally by writing
        \[
            \phi(n) = \sum_{k \leq n \text{ and } (k,n) = 1} 1 = \sum_{k \leq
            n} \sum_{d \mid (k,n)} \mu(d)
        \]
        and reordering sums.
\end{itemize}

\end{document}
