\documentclass[a4paper,11pt]{article}
\pagenumbering{arabic}
\usepackage{../environment}

\author{Max von Consbruch}

\begin{document}

\begin{center}
    \huge{Solution to Sheet 5.}
\end{center}

\section*{Problem 1}
We have basically solved this already on sheet 3. Note that as $d_1 \mid q_1$ and 
$d_2 \mid q_2$, we have $(d_1, d_2) = 1$, so (by sheet 3) there are primitive
characters $\psi_i$ mod $d_i$ with $\chi_i = \psi_i \chi_{0,q_i}$ (here again
$\chi_{0,q_i}$ is the principal character mod $q_i$) whose product $\psi =
\psi_1 \psi_2$ is a primitive character mod $d_1d_2$. Modulo $q$, this
reveals
\[
    \chi_1 \chi_2 =  (\chi_{0, q_1} \psi_1) (\chi_{0, q_2} \psi_2) =
    \chi_{0,q_1 q_2} \psi, 
\]
hence $\psi$ is induced by a primitive character mod $d_1d_2$. 

It is easily seen that the coprimality condition is necessary. Take any real
character $\chi$ mod $q$ for example, then $\chi^2 = 1$ and has conducter 
$1 \neq q$. 

\section*{Problem 2}
We have to show the bound
\[
    \int_{(-A + \tfrac 12)} \Gamma(s) x^s \dc s \ll \frac{x^{-A +
    1/2}}{(A-1)!}.
\]
Note that the integral exists by the rapid decay of $\Gamma$ along vertical
lines. However, we cannot apply Stirling's formula to bound the integral directly
as Stirling a priori only gives uniform bounds in regions of the form $\abs{\arg(s)
- \pi} \geq \delta > 0$. We can however apply stirlings formula if we apply the recurrence
$s\Gamma(s) = \Gamma(s+1)$ repeatedly: 
\begin{multline*}
    \int_{(-A+1/2)} \Gamma(s) x^s \dc s \ll 
    \int_{(-A+1/2)} \abs{\Gamma(s) x^s} \dc s \ll 
    x^{-A+1/2} \int_{(1/2)} \abs{\Gamma(s-A)} \dc s \\ 
    = x^{-A+1/2} \int_{(1/2)} \abs{\frac {\Gamma(s)}{(s-A+1) \cdots (s-1)}}
    \dc s \leq \frac{x^{-A + 1/2}}{(A-1)!} \int_{(1/2)} \abs{\Gamma(s)} \dc s. 
\end{multline*}

\textbf{Notes.}
Once we know this inequality, we actually can do better: Remember that $\Gamma$ has poles at
the negative integers, the residue at $-n$ is given by $\tfrac{(-1)^n}{n!}$. Hence for 
(large) $T > 0$, we have that 
\[
    \int_{1/2-A-\ic T}^{1/2 - A +\ic T} \Gamma(s) x^s \dc s =
    2 \pi \ic \frac{(-x)^{-A}}{A!} + \int_{1/2-A-\ic T}^{-1/2 - A + \ic T} \Gamma(s) x^s
    \dc s + O\left( \int_{1/2-A-\ic T}^{-1/2-A - \ic T}  \Gamma(s) x^s \dc s\right). 
\]
By the rapid decay of $\Gamma$, the horizontal integral vanishes as $T
\rightarrow \infty$, and we can bound the vertical integral using what we
showed before, applied to $A+1$. This yields
\[
    \int_{(-A+1/2)} \Gamma(s) x^s \dc s = 2 \pi \ic \frac{(-x)^{-A}}{A!} + O\left(\frac{x^{-A-1/2}}{A!}\right).
\]
In fact, as for every $x>0$ the fraction $x^A/A!$ tends to zero as $A \to \infty$, we
may repeat this as often as we want, obtaining
\[
    \frac 1{2 \pi \ic} \int_{(-A+1/2)} \Gamma(s) x^s \dc s = \sum_{k=A}^\infty \frac{(-x)^{-k}}{k!} 
    = \ec^{-\frac 1x} - \sum_{k=0}^{A-1} \frac{(-x)^{-k}}{k!}.
\]
The equation for $A= 0$ is nothing new! As $\Gamma(s)$ is holomorphic
for $\Re s > 0$ we already know that $\cM(e^{-x})(s) = \Gamma(s)$, so
\[
    \ec^{-x} = \frac{1}{2 \pi \ic} \int_{(1/2)} \cM(\ec^{-x})(s) x^{-s} \dc s
    = \frac{1}{2 \pi \ic} \int_{(1/2)} \Gamma(s) x^{-s} \dc s.
\]
Now replace $x$ by $x^{-1}$. 

\section*{Problem 3}
We substitute $p^{-s} = x$ to find the equivalent
\[
    \sum_{k=0}^\infty \beta(k) x^k = \frac 1 {P(x)}. 
\]
Where $P(x) = \prod_{j=1}^d(1-\alpha_j x)= \sum_{i = 0}^d a_i x^i$ (in particular,
$a_0 = 1$). Multiply both sides with $P$, revealing
\[
    \sum_{d=0}^\infty x^d \sum_{k=0}^d \beta(d-k) a_k = 1.
\]
Equating coeffiecients gives that for $k>0$, 
\[
    \sum_{k=0}^d a_k \beta(d-k) = 0,
\]
which, after subtracting $\beta(d)$ on both sides and setting $c_i = -a_{i+1}$,
gives the desired recurrence.

\contactend
\end{document}
