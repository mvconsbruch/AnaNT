\documentclass[a4paper,11pt]{article}
\pagenumbering{arabic}
\usepackage{../environment}

\author{Max von Consbruch}

\begin{document}

\begin{center}
    \huge{Approximate functional equation, what is it all about?!}
\end{center}

\section{A classical approximate functional equation} % (fold)
\label{sec:A classical approximate functional equation}

The aim of the approximate functional equation is to understand $\zeta$ (or more 
generally, any $L$-function) better in the critical strip $0 < \Re s < 1$. 
In this first part, we will focus on the case of $\zeta$. 
When we proved that $\zeta$ has a meromorphic continuation to $\Re s > 0$, we 
used partial summation on the dirichlet series, showing that for $\Re s > 1$, we have
\[
    \zeta(s) = \sum_{n=1}^\infty n^{-s} = \frac{s}{s-1} - s \int_1^\infty \{t\} t^{-s-1} \dc t,
\]
where the RHS is defined also for $\Re s > 0$. 

Given some $N > 0$, a similar expression arises if we use partial
summation on the truncated Dirichlet series, as
\[
    \sum_{n \leq N} n^{-s} = N^{1-s} + s \int_1^N \lfloor t \rfloor t^{-s-1} \dc t
    = \frac {N^{1-s}} {1-s} + \frac s{s-1} - s \int_1^N \{t\} t^{-s-1} \dc t.
\]
Now one might be tempted to comare the RHSs of the previous two equations. Writing 
$s = \sigma + \ic t$, we find
\[
    \zeta(s) - \sum_{n \leq N}n^{-s} = \frac{N^{1-s}}{s-1} - s \int_1^N \{t\}
    t^{-s-1} \dc t = \frac {N^{1-s}}{s-1} + O\left(\frac{\abs{s}}\sigma N^{-\sigma}
        \right).
\]
The important observation is that nothing goes wrong if we pass from $\Re s > 1$
to $\Re s > 0$! We found that $\zeta$ is approximated by the first terms 
in its dirichlet series, even in the critical strip. As is turns out, this 
approximation is not great, as we still have that annoying $\abs s$ in the $O$-term,
which forces us to choose $N$ large (roughly like $t^{1/\sigma}$) to make use
of this approximation. The crucial thing we missed in our approximation is that
$n^{it} = \ec^{(\log n) \ic t}$ oscillates and constitutes a lot of
cancellation. Using some sort of approximate fourier transform called 
\textit{van der Corput summation}, one can get hold of this oscillation to
obtain a stronger bound on the error, given by
\[
    \zeta(s) = \sum_{n \leq x} n^{-s} - \frac{x^{1-s}}{1-s} + O(x^{-\sigma}).
\]
This is uniform in $\sigma > \sigma_0$ once we fix $\sigma_0 > 0$, provided that
$\abs t \leq 4x$. (Take a look in Chapter 4 of Brüdern's book for details).

Choosing $\sigma = \frac 12$, we find that 
\[
    \zeta(s) \ll \sum_{n \leq t} n^{-\sigma} + t^{1-\sigma} \ll t^{1- \sigma},
\]
which is an okay bound, but not as good as we'd like. The convexity bound already 
gave that $\zeta(s) \ll t^{\frac{1-\sigma}2 + \varepsilon}$, so we could hope 
that we could do even better, approximating $\zeta$ with sums of length $\sqrt t$.
Unfortunately, it does not seem as if such a identity holds true. 

However, we can apply the functional equation to obtain a similar approximation
of $\zeta$, just from the other side (i.e., at $1-s$). Even better, we might be
able to combine these approximations to obtain a better approximation of 
$\zeta$. This is the Idea of the \textit{approximate functional equation}. 
And indeed, it gives what we hoped for: We can essentially approximate $\zeta$ by
Dirichlet-sums of length $\sqrt t$. If we write $\zeta(s) = \Delta(s) \zeta(1-s)$,
the theorem reads as
\begin{thm}[Approximate functional equation]
    Let $0 < \sigma < 1$ and $2 \pi xy = t$ , where $x,y > 1$. Then
    \[
    \zeta(s) = \sum_{n \leq x}n^{-s} + \Delta(s) \sum_{n \leq y} n^{s-1} 
    + O((x^{-\sigma} + t^{1/2-\sigma} y^{\sigma-1}) \log t).
    \]
\end{thm}
We shouldn't worry about the shape of the error term too much, just observe that 
the balanced case is given when $\sigma = \frac 12$ and $x=y= \sqrt{\frac x
{2\pi}}.$ 

This AFC is stronger than the one we had in the lecture. For example, it allows
us to deduce asymptotic formulas for the second and fourth moments of $\zeta$
on the critical line, only using elementary manipulations. We get
\[
    \int_0^T \abs{\zeta(\tfrac 12 + \ic t)}^2 \dc t = T \log T + O(T)
\]
and 
\[
    \int_0^T \abs{\zeta(\tfrac 12 + \ic t)}^4 \dc t = \frac{T (\log T)^4}{2\pi} + 
    O(T(\log T)^3).
\]
(Again, you can read this up in Brüdern's book).


% section A classical approximate functional equation

\section{The smoothed approximate functional equation} % (fold)
\label{sec:Our approximate functional equation}
You may wonder, how does the approximate functional equation from before relate
to the one we had in the lecture?! It looked much more complicated and did not
allow us to deduce asymptotic formulas. In the lecture, we proved a smoothed
version of the formula above, and usually, smoothed formulas are easier to
prove, but harder to use. The reason for that phenomenon is that 
for smooth, compactly supported functions, the mellin tranform has rapid
decay along vertical lines. This makes calculations really easy.

% section Our approximate functional equation



\end{document}
