\documentclass[a4paper,11pt]{article}
\pagenumbering{arabic}
\usepackage{../environment}

\author{Max von Consbruch}

\begin{document}

\begin{center}
    \huge{Solution to Sheet 4.}
\end{center}

\section*{Problem 1} % (fold)
\textbf{a)} Let $g(x) = f(qx + a)$, so that
\[
    \sum_{n \equiv a \text{ (mod $q$)}} f(n) = \sum_{m \in \Z} g(m).
\]
We want to apply Poisson summation to $g$. The results of lemma (2.3) directly
give that 
$$\hat g(y) = \frac 1q e\left( \frac{ya}q \right) \hat f\left(\frac yq \right).$$
The claim follows, as
$$ \sum_{m \in \Z} g(m) = \sum_{m \in \Z} \hat g(m) = \frac 1q \sum_{m \in \Z} 
    e\left( \frac{ma}q \right) \hat f\left(\frac mq \right).$$

\textbf{b)} We would like to apply Poisson summation again, however we cannot 
calculate the "Fourier transform" of $f\chi$, as, $\chi$ is only defined 
on integers. We can abuse that $\chi$ is periodic though, rewriting
\[
    \sum_{m \in \Z} f(m) \chi(m) = \sum_{a \text{ (mod $q$)}} \chi(a) \sum_{m \equiv a \text{ (mod $q$)}}
    f(m).
\]
Applying Poisson summation to the inner sum (we already did this in part
a)) gives
\[
    \sum_{m \in \Z} f(m) \chi(m) = \frac 1q \sum_{a \text{ (mod $q$)}} \chi(a) \sum_{m \in \Z} 
    e\left( \frac{ma}q \right) \hat f\left(\frac mq \right).
\]
Reordering sums, we obtain
\begin{multline*}
     \frac 1q \sum_{a \text{ (mod $q$)}} \chi(a) \sum_{m \in \Z} 
     e\left( \frac{ma}q \right) \hat f\left(\frac mq \right) = \frac 1q \sum_{m
         \in \Z} \hat f\left(\frac mq \right) \left( \sum_{a \text{ (mod $q$)}}
             \chi(a) e\left(\frac{ma}q \right) \right) \\ =  \frac 1q \sum_{m
         \in \Z} \hat f\left(\frac mq \right) \tau(\chi) \overline \chi(m) =
         \frac {\tau(\chi)}q \sum_{m \in \Z} \hat f\left(\frac mq \right)
         \overline \chi(m). 
\end{multline*}

\textbf{Notes after correcting.}
\begin{itemize}
    \item In part a), instead of using the results from the lecture, we can
        also obtain the formula for the fourier transform directly.
        Setting $g(x) = f(qx + a)$ and substituting $u = qx + a$, we obtain
        \[
            \hat g ( y ) = 
            \int_\R f(qx+a) e(-xy) \dc x = 
            \frac 1q \int_\R f(u) e(-u\tfrac yq + \tfrac{ay}q)
            \dc u = \frac 1q e(\tfrac {ay}q) \hat f(\tfrac yq).
        \]
\end{itemize}

\section*{Problem 2}
We do as the hint commands. Let 
\[
    f(t) = 
    \begin{cases}
        \ec^{-1/t^2} \quad &t>0 \\
        0 \quad & \text{else.}
    \end{cases}
\]
Then one easily checks that $f$ is smooth and non-negative. Now we put 
$g(t) = \frac{f(t)}{f(t) + f(1-t)}$, which is still smooth and non-negative. 
We clearly have $g(t) = 0$ if $t<0$, $g(t) \in [0,1]$ for $t \in [0,1]$ and 
$g(t) = 1$ for $t > 1$. Finally, define 
\[
    h(t) = g \left( \frac{t-X+Z}Z \right) - g \left( \frac{t-X-Y}Z \right).
\]
This satisfies $\textnormal{supp}(h) \subset [X-Z, X+Y+Z]$ and 
$h(t) = 1$ for $t \in [X,X+Y]$. 
We still need to check that $\norm{f^{(j)}}_1 \ll Z^{1-j}$ for all $j \in \N$.
One could expect this to be really messy as calculating the higher derivatives
of $h$ seems horrible. However, we just need that the $j$-th derivative of $h$ 
is given by
\[
    h^{(j)}(t) = Z^{-j} \left( g^{(j)} \left( \tfrac{t-X+Z}Z \right) - g^{(j)}
    \left( \tfrac{t-X-Y}Z \right) \right).
\]
As $h^{(j)}$ vanishes everywhere except $[X-Z,X]$ and $[X+Y, X+Y+Z]$, we obtain
by a linear change of variables 
\[
    \norm{h^{(j)}}_1 = \left(\int_{X+Z}^X + \int_{X+Y}^{X+Y+Z}\right)
    \abs{h^{(j)}(t)} \dc t = 2Z^{1-j} \int_0^1 \abs{g^{(j)}(t)} \dc t \ll_j Z^{1-j}.
\]


\section*{Problem 3}
As the hint commands, we apply partial summation to the definition of
$\tau(\chi)$, obtaining
\[
    \abs{\tau(\chi)} = \sum_{h=1}^q \chi(h) e(h/q) = e(q/q) \sum_{h=1}^q
    \chi(h) - \frac{2\pi \ic}{q} \int_1^q e(t/q) \sum_{h \leq t} \chi(h) 
    \dc t.
\]
As $\chi \neq \chi_0$, the sum $\sum_{h=1}^q \chi(h)$ vanishes. We also
know by theorem (1.23) that $\abs{\tau(\chi)} = \sqrt q$. 
Let $M$ deonte the supremum of the absolute values of $\sum_{h \leq x}
\chi(h)$ for varying $x$ (By Polya-Vinogradov, $M < \infty$). Then we obtain
\[
    \frac{q^{3/2}}{2 \pi} = \abs{\int_1^q e(t/q) \sum_{h \leq t} \chi(h) \dc t} \leq \int_1^q \abs{ \sum_{h \leq t}  \chi(h) } \dc t \leq (q-1)M,
\]
which is even a tad stronger than what we had to show. 

\section*{Problem 4}
Let's just plug in the definition and look at what we have here.
\[
    \tau(\chi_1 \chi_2) = \sum_{h \ (q)} \chi_1(h) \chi_2(h) e(h/q),
\]
where $q = q_1 q_2$. By the chinese remainder theorem, taking residues mod $q$ gives
a bijection
\[
    \{h_1 q_2 + h_2 q_1 \mid 1 \leq h_i \leq q_i\} \to \Z/q\Z.
\]
Thus we may rewrite the sum above as 
\[
    \tau(\chi_1 \chi_2) = \sum_{1 \leq h_1 \leq q_1} \sum_{1 \leq h_2 \leq q_2}\chi_1(h_1q_2 + h_2q_1) \chi_2(h_1q_2 + h_2 q_1) e(\tfrac{h_1q_2+h_2q_1}q),
\]
and the claim follows after a few manipulations:
\begin{multline*}
     \sum_{1 \leq h_1 \leq q_1} \sum_{1 \leq h_2 \leq q_2}\chi_1(h_1q_2 + h_2q_1) \chi_2(h_1q_2
     + h_2 q_1) e(\tfrac{h_1q_2+h_2q_1}q) \\
     = \sum_{1 \leq h_1 \leq q_1} \sum_{1 \leq h_2 \leq q_2}\chi_1(h_1q_2) \chi_2(h_2 q_1)
     e(\tfrac{h_1q_2}q) e(\tfrac{h_2q_1}q) \\ 
     = \left(\chi_1(q_2)\sum_{1 \leq h_1 \leq q_1}
     \chi_1(q_2)e(\tfrac{h_1}{q_1})\right) \left(\chi_2(q_1)\sum_{1 \leq h_2 \leq q_2}
     \chi_2(q_1)e(\tfrac{h_2}{q_2})\right) = \chi_1(q_2)\tau(\chi_1) \chi_2(q_1)\tau(\chi_2).
\end{multline*}


\end{document}
